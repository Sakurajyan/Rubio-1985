\documentclass{report}
\usepackage{amsthm}
\usepackage{amsmath}
\usepackage{amsfonts}
\usepackage{xcolor}
\usepackage{mathrsfs}
\newtheorem*{remark}{Remark}
\newtheorem{theorem}{Theorem}
\newtheorem{definition}[theorem]{Definition}
\newtheorem{proposition}[theorem]{Proposition}
\newtheorem{corollary}[theorem]{Corollary}
\numberwithin{theorem}{subsection}
\newcommand{\norm}[1]{\left\lVert#1\right\rVert}
\newcommand{\abs}[1]{\left\lvert#1\right\rvert}
\renewcommand{\Re}{\operatorname{Re}}
\renewcommand{\Im}{\operatorname{Im}}
\begin{document}
\section{The Conjugate Function}
In section 1, given a integrable function $f$, we know the harmonic function $u=P(f)$ determines a holomorphic function $F$ up to a constant $c$ if we consider $u$
as the real part of $F$. Let $v(re^{it})=F(re^{it})$ with $v(0)=0$. Then we can define the conjugate function of $f$ to be:
\begin{equation*}
    \tilde{f}(t)=\lim_{t\to 1}{v(re^{it})}
\end{equation*}
The next theorem ensures the above limit exists for a.e. t:
\begin{theorem}[theorem 5.2 in book]
    Let $F\in H(D)$ be such that $\Re{F(z)}\geq 0$ for every $z\in D$. Then $F$ has N.T. limits at almost every boundary point.
\end{theorem}
\begin{remark}[notes on proof of theorem 5.2 in book]
    
\end{remark}
\subsection{Estimate conjugate function $\tilde{f}$ for $f\in L^p$, $1<p<\infty$}
\subsection{Estimate conjugate function $\tilde{f}$ for $f\in L^1$}
\subsection{Conjugate function and $H^p$ space}
\subsection{Conjugate function operator}
\end{document}
