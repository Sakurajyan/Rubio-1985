% \documentclass{report}
% \usepackage{amsthm}
% \usepackage{amsmath}
% \usepackage{amsfonts}
% \usepackage{xcolor}
% \usepackage{mathrsfs}
% \newtheorem*{remark}{Remark}
% \newtheorem{theorem}{Theorem}
% \newtheorem{definition}[theorem]{Definition}
% \newtheorem{proposition}[theorem]{Proposition}
% \numberwithin{theorem}{subsection}
% \newcommand{\norm}[1]{\left\lVert#1\right\rVert}
% \newcommand{\abs}[1]{\left\lvert#1\right\rvert}
% \renewcommand{\Re}{\operatorname{Re}}
% \renewcommand{\Im}{\operatorname{Im}}
% \begin{document}
\section{Some Classical Inequalities}
In this section we study two classical Inequalities: Hardy's inequality and Fejer-Riesz inequality. The first inequality is an
example of why $H^p$ is a natural replacement of $L^p$ for $p\leq 1$. The second inequality shows some geometry properties of conformal mappings.
\subsection{Hardy's inequality}
\begin{theorem}[Hardy's inequality]\label{Hardy's inequality}
    Let $F(z)=\sum_{j=0}^{\infty}{a_jz^j}$ be in $H^1$. Then:
    \begin{equation*}
        \sum_{j=0}^\infty{\frac{\abs{a_j}}{j+1}}\leq C\norm{F}_{H^1}
    \end{equation*}
    with a constant $C$ independent of $F$.
\end{theorem}
\begin{remark}[notes on proof of theorem \ref{Hardy's inequality}]
    We know the principal branch of the logarithm $\log{z}=\log{r}+i\theta$ where $z=re^{i\theta}$ with $\abs{\theta}<\pi$. Thus
    $\Im{\log{1-z}}=\arg{1-z}$. It is easy to see $-\frac{\pi}{2}<\arg{1-z}<\frac{\pi}{2}$.\par
    \begin{align*}
        F(re^{it})u(re^{it})= & (\sum_{j=0}^{\infty}{a_j(re^{it})^j})(\frac{i}{2}\sum_{j\neq 0}{j^{-1}r^{\abs{j}}e^{i j t}})    \\
        =                     & (\sum_{j=0}^{\infty}{a_j r^j e^{i j t}})(\frac{i}{2}\sum_{k\neq 0}{k^{-1}r^{\abs{k}}e^{i k t}})
    \end{align*}
    After taking integral, only $j+k=0$ term does not vanish, thus:
    \begin{align*}
        \frac{1}{2\pi}\int_{-\pi}^{\pi}F(re^{it})u(re^{it})d t= & (\frac{i}{2}\sum_{j+k=0}\frac{1}{2\pi}\int_{-\pi}^{\pi}{a_j r^j e^{i j t}k^{-1}r^{\abs{k}}e^{i k t}}d t) \\
        =                                                       & \frac{i}{2}\sum_{j+k=0}\frac{1}{2\pi}\int_{-\pi}^{\pi}{a_j r^{j+\abs{k}}e^{i(j+k)t}k^{-1}}d t            \\
        =                                                       & \frac{i}{2}\sum_{j=1}^{\infty}\frac{1}{2\pi}\int_{-\pi}^{\pi}{a_j r^{2j}(-j)^{-1}}d t                    \\
        =                                                       & \frac{i}{2}\sum_{j=1}^{\infty}{a_j r^{2j}(-j)^{-1}}                                                      \\
        =                                                       & -\frac{i}{2}\sum_{j=1}^{\infty}{a_j j^{-1}{\color{red}r^{2j}}}
    \end{align*}
\end{remark}
The corollary 4.2 in book shows that if $F(e^{it})$ is absolutely continuous (equivalent to $F'\in H^1$), then $(\hat{F}(n))_n\in\ell^1$. But the converse is not true. $(\hat{F}(n))_n\in\ell^1$ only implies $F$ extends to a continuous function on $\bar{D}$

\begin{remark}[Errata of $\Re{H^1}$]
    Let $g(t)$ be $\Re{F(e^{it})}=\sum_{j\geq 0}{a_je^{i j t}}$. Then
    \begin{align*}
        g(t)= & \frac{F(e^{it})+\overline{F(e^{it})}}{2}                                                                       \\
        =     & \frac{a_0+\bar{a_0}}{2}+\sum_{j\geq 0}{\frac{a_j}{2}e^{i j t}}+\sum_{j\geq 0}{\frac{\bar{a_j}}{2}e^{-i j t}}   \\
        =     & \frac{a_0+\bar{a_0}}{2}+\sum_{j> 0}{\frac{a_j}{2} e^{i j t}}+\sum_{j< 0}{\frac{\overline{a_{-j}}}{2}e^{i j t}} \\
        =     & \frac{a_0+\bar{a_0}}{2}+\sum_{j\neq 0}{\hat{g}(j) e^{i j t}}
    \end{align*}
    {\color{red}where $\hat{g}(j)=\frac{a_j}{2}$ for $j>0$, $\hat{g}(j)=\frac{\overline{a_{-j}}}{2}$ for $j>0$ and $\hat{g}(j)=\Re{a_0}$}.
    Thus $\abs{a_j}=\abs{\hat{g}(j)}+\abs{\hat{g}(-j)}$. Substitute $\abs{a_j}$ to $\sum_{j=1}^\infty{\frac{\abs{a_j}}{j}}\leq \pi\norm{F}_{H^1}$.
    We have $\sum_{j\neq 0}{\abs{\frac{\hat{f}(j)}{j}}}\leq \pi\norm{f}_{\Re{H^1}}$
\end{remark}
We have $\Re{H^1}$ is a proper subspace of $\Re{L^1}$. And Hardy's inequality may be considered an extension to $p=1$ of Paley's
inequality which says that for $f\in L^p$ with $1<p\leq 2$:
\begin{equation*}
    \sum_{j=-\infty}^\infty{\frac{\abs{\hat{f}(j)}^p}{\abs{j}^{p-2}}}\leq C_p\norm{F}_p^p
\end{equation*}
Later we will see in $\mathbb{R}^n$ this inequality can be extended to $H^p$ for $p<1$. And $H^p$ for $p\leq 1$ are natural substitutes of
Lebesgue spaces $L^p$.
\subsection{Fejer-Riesz inequality}
Recall the final corollary in last section. Let $F$ be a conformal map from $D$ to interior domain bounded by a Jordan curve $\Gamma$. Then $\Gamma$ is rectifiable if and only if $F'\in H^1$.
\begin{theorem}[Fejer-Riesz inequality]
    Let $F\in H^p$, $0<p<\infty$, then
    \begin{equation*}
        \int_{-1}^{1}{\abs{F(x)}^p dx}\leq \frac{1}{2}\int_{-\pi}^{\pi}{\abs{F(e^{it})}^p d t}
    \end{equation*}
\end{theorem}\par
To prove this theorem, we first prove the $p=2$ case. Then for $p\neq 2$ case, we factorize $F(z)=B(z)H(z)$ and let $\abs{G(z)}^2=\abs{H(z)}^p$
to reduce this case to $p=2$.\par
Here is a direct application of this inequality. Let $F$ be the conformal map from $D$ to interior domain bounded by a Jordan curve $\Gamma$. Then image of diameter of $D$ has length at most half of length of $\Gamma$ (corollary 4.6 in book).\par
\begin{remark}[notes on proof of corollary 4.6 in book]
    To prove that $\frac{1}{2}$ is the best constant in corollary 4.6 in book, we only need to show there is a conformal map
    from $D$ to interior domain bounded by a rectifiable Jordan curve $\Gamma$, the constant $\frac{1}{2}$ can not be smaller. Let $F(z)$ is
    a conformal map from $D$ to rectangle $\{x+iy:\abs{x}<1,\abs{y}<\epsilon\}$ and $F$ maps segment $(-1,1)$ in $D$ to segment $(-1,1)$
    in rectangle. It is easy to construct this map. The constant has to be at least $\frac{2}{4+4\epsilon}$. Let $\epsilon\to 0$ we conclude
    $\frac{1}{2}$ is the best constant.
\end{remark}
Another usage of conformal mapping $F'\in H^1$ is following: $F$ can be extended on $\bar{D}$ and $F$ is still conformal. More precisely,
Let $F$ be a conformal mapping from $D$ to interior domain bounded by a rectifiable Jordan curve $\Gamma$. $F$ is also conformal at
almost every boundary point (corollary 4.7 in book).
\begin{remark}[notes on proof of corollary 4.7 in book]
    The step:
    \begin{equation*}
        \frac{F(e^{it_0})-F(z)}{e^{it_0}-z}-F'(e^{it_0})= \frac{1}{e^{it_0}-z}\int_z^{e^{it_0}}{(F'(\xi)-F'(e^{it_0}))d\xi}\to 0
    \end{equation*}
    as $z\to e^{it_0}$ N.T. is {\color{blue}by mean value theorem of integration}.\par
    {\color{blue}I don't know why the tangent to $\Gamma$ at the point $F(e^{it_0})$ happens for a.e. boundary point $e^{it_0}$}.\par
    The angle between $\gamma$ and boundary in $D$ is $\lim{\arg{z-e^{it_0}}}-t_0-\frac{\pi}{2}$ and
    The angle between $F(\gamma)$ and boundary in $F(D)$ is $\lim{\arg{F(z)-F(e^{it_0})}}-t_0-\arg{(\frac{d}{dt}(F(e^{it}))\vert_{t=t_0})}$.
    Since $F$ is conformal in $D$, to prove $F$ is conformal in $\bar{D}$, we only need to prove the conformal map preserves angle on boundary.
    That is:
    \begin{equation*}
        \lim_{z\to e^{it_0}}{\arg{(z-e^{it_0}})}-t_0-\frac{\pi}{2}=\lim_{z\to e^{it_0}}{\arg{(F(z)-F(e^{it_0}))}}-\arg{(\frac{d}{dt}(F(e^{it}))\vert_{t=t_0})}
    \end{equation*}
    We have $\frac{d}{dt}F(e^{it})\vert_{t=t_0}=ie^{it_0}F'(e^{it_0})$. Thus $\arg{(\frac{d}{dt}(F(e^{it}))\vert_{t=t_0})}=\frac{\pi}{2}+t_0+\arg{F'(e^{it_0})}$. So the equality is the same as:
    \begin{equation*}
        \lim_{z\to e^{it_0}}{\arg{(z-e^{it_0}})}=\lim_{z\to e^{it_0}}{\arg{(F(z)-F(e^{it_0}))}}-\arg{F'(e^{it_0})}
    \end{equation*}
    which is clearly if we take $\arg$ in both sides in $\lim_{z\to e^{it_0}}\frac{F(e^{it_0})-F(z)}{e^{it_0}-z}=F'(e^{it_0})$.
    We use $t_0+\frac{\pi}{2}$ instead of $t_0-\frac{\pi}{2}$ match to $\arg{(\frac{d}{dt}(F(e^{it}))\vert_{t=t_0})}$ since they are in the same direction.
\end{remark}
% \end{document}

