% \documentclass{report}
% \usepackage{amsthm}
% \usepackage{amsmath}
% \usepackage{amsfonts}
% \usepackage{xcolor}
% \newtheorem*{remark}{Remark}
% \newtheorem*{definition}{Definition}
% \newtheorem{theorem}{Theorem}
% \numberwithin{theorem}{subsection}
% \newcommand{\norm}[1]{\left\lVert#1\right\rVert}
% \newcommand{\abs}[1]{\left\lvert#1\right\rvert}
% \begin{document}
\chapter{Classical Theory of Hardy Space}
% In this chapter we will talk about the connection between properties of harmonic or analytic function in the unit disc
% $D$ of the complex plane and the Fourier Analysis of their boundary values in the torus $T=\partial D$.\par
% The two important tools are Riesz factorization and Szego factorization. These tools let us analyze function in space 
% similar with $L^p$ for $p<1$. In Hardy space, we talk about an important element, conjugate function. We will also 
% meet our first example of weighted norm inequality: Helson-Szego theorem.
\section{Harmonic Functions, Poisson Representation}
In plane case, we can rewrite harmonic condition: $\Delta F=0$ as
$\Delta F=(\frac{\partial}{\partial x}-i\frac{\partial}{\partial y})(\frac{\partial}{\partial x}+i\frac{\partial}{\partial y})F=0$. Notices the equation $\frac{\partial}{\partial x}+i\frac{\partial}{\partial y}=0$ is equivalent to the Cauchy-Riemann equations for function $F=u+iv$.\par
\begin{remark}[holomorphic indicates harmonic]
    If function F satisfies C-R equations, then F satisfies Laplace equation for $\mathbb{C}$. And if F satisfies $\Delta=0$, so does $\bar{F}$. Thus $F$ and $\bar{F}$ are all harmonic functions.
    Besides, the real and imaginary part of F, u and v also satisfy $\Delta=0$, means u and v are harmonic functions.
\end{remark}
\begin{remark}[harmonic indicates holomorphic]
    Assume that u satisfies $\Delta=0$. If we let $v_x=-u_y$ and $v_y=u_x$, then we have $v_{xy}=v_{yx}$, which indicates exists of v (equivalent of
    Fubini's Theorem and the equality of the mixed partial derivatives). And $v_{xx}+v_{yy}=-u_{yx}+u_{xy}=0$ indicates $v$ is also
    harmonic. $v$ is determined up to an additive constant, and $F=u+iv$ is holomorphic.
\end{remark}
\begin{remark}
    Later we will see holomorphic function is the special case of harmonic function.
\end{remark}
If $u$ is a real harmonic function, by above remark, we know $u=\operatorname{Re}{F}$ for some holomorphic function $F(z)=\sum_{k=0}^{\infty}{c_k z^k}$. Then we can derive the series representation of $u$:
\begin{equation}\label{series of harmonic function}
    u(re^{i\theta})=\sum_{k=-\infty}^\infty{a_kr^{\abs{k}}e^{i k\theta}}
\end{equation}
and {\color{blue}it converges uniformly on compact subsets of $D(0,R)$}.
\begin{remark}[uniformly converge on compact subsets]
    Using mean value property, sequence of holomorphic functions which uniformly converge on compact subsets, the limit is a holomorphic function. \par
    The partial sum of $u_n(re^{i\theta})=\sum_{k=-n}^n{a_kr^{\abs{k}}e^{i k\theta}}$ is obviously harmonic, and it converges uniformly on compact subsets of $D(0,R)$. Thus $u(re^{i\theta})$ is harmonic.
\end{remark}\par

\subsection{Harmonic function to Poisson integral (or Poisson representation of harmonic function)}\label{harmonic function can be written as Poisson integral}
Assumed $u$ is harmonic in $D(0,R). $If $R>1$, we can represent $a_k$ in equation \eqref{series of harmonic function} by $u(e^{i t})$ using Fourier analysis. Finally we derive the Poisson kernel and the Poisson representation:
\begin{equation}
    u(re^{i\theta})=\frac{1}{2\pi}\int_{-\pi}^\pi{P_r(\theta-t)u(e^{i t})}d t\label{Poisson integral 1}
\end{equation}
If $u$ is only harmonic on $D(0,1)$, the equation \eqref{Poisson integral 1} still can be valid in some sense once we add some restriction on $u$.
\begin{theorem}[theorem 1.3 in book]\label{Poisson representation}
    Let $u$ be a harmonic function in $D$ such that
    \begin{equation}\label{Poisson representation condition}
        \sup_{0\leq r<1}{\int_{-\pi}^{\pi}\abs{u(re^{i t})}^p d t}<\infty
    \end{equation}
    for some $p>1$. Then there is a function $f\in L^p$ such that
    \begin{equation}
        u(re^{i\theta})=\frac{1}{2\pi}\int_{-\pi}^\pi{P_r(\theta-t)f(t)}d t
    \end{equation}
\end{theorem}
This theorem still holds for $p=\infty$ if we replace inequality \eqref{Poisson representation condition} by \\$\sup_{0\leq r<1}\abs{u(re^{i t})}<\infty$.\par
In later section, we will see left side of inequality \eqref{Poisson representation condition} is $H^p$ norm. The proof of Theorem \ref{Poisson representation} shows
that how dual space, w*-topology and Banach-Alaoglu theorem perform in integral representation. The tricky part is using representation $u(r_nre^{i\theta})$.
In the proof we also need $P_r(\theta)\in L^{p'}$ which is easy since  $\norm{P_r(\theta)}_\infty=\frac{1+r}{1-r}$ and $L^\infty([-\pi,\pi]) \subset L^{p'}([-\pi,\pi])$.\par

\begin{remark}[some details of proof of theorem \ref{Poisson representation}]
    In metrizable space, compact and sequentially compact are equivalent.\par
    Let $f_n(t)=u(r_ne^{i t})$, $(f_n)\subset L^{p'*}$ is in a close ball w.r.t p-norm and Banach-Alaoglu theorem indicates this ball is w* compact. Since w*-compact is equivalent
    to w*-sequentially compact if this close ball is metrizable in w*-topology, we need to prove this close ball is metrizable in w*-topology.\par
    Here is a direct proof. Since $L^{p'}$ is separable there is a countable dense set $(g_n)\in L^{p'}$. For every $g\in L^{p'}$, let $\hat{g}(f)=f(g)$ where $f\in L^{p'*}$. Any pair of points $f_1,f_2\in L^{p'*}$ can be separate
    by some $g_i\in L^{p'}$ (since every $\hat{g_n}$ is w*-continuous, density of $(g_n)$ ensures $f_1,f_2$ can be separate),
    Thus $(\hat{g_n})$ is a countable family of continuous functions that separates points in $L^{p'*}$.
    Then we can use metric: $d(f_1,f_2)=\sum_{n=0}^\infty{2^{-n}\abs{f_1(g_n)-f_2(g_n)}}$.\par
    {\color{blue}We have proved that $L^{p'*}$ is metrizable in w*-topology. Here is a topological thought.  By Nagata-Smirnov metrization theorem, $L^{p'*}$ is regular. we don't know how to shown this property directly. Also, $L^{p'*}$ a basis countably locally finite (by Nagata-Smirnov metrization theorem), we also want to show this property directly. Besides, consider Urysohn metrization theorem, if we have shown $L^{p'*}$ is regular, $L^{p'*}$ is metrizable once we show there is a countable basis for $L^{p'*}$. We don't know if it is possible}.\par
    Now $(f_n)$ is w* sequentially compact in $L^{p'*}$. This means for any $g\in L^{p'}$, there is a subsequence $(f_{n_k})$
    and $f\in L^{p'*}$ such that $\int gf_{n_k}\to\int gf$.
\end{remark}
\begin{remark}
    $L^{p'*}\cong L^p$. Thus $L^{p'*}$ is a normed space and $L^{p'*}$ is metrizable.
\end{remark}
The $p=1$ case is failed since we can not use w*-topology. $L^1$ is not a separate dual space of some space (Assume $L^1=X^*$. $L^1$ is separable in w*-topology implies $X$ is separable).
    {\color{blue} One method showing $L^1$ is not  a separate dual space of some space by showing $L^1$ does not have the Radon-Nikodym property (Radon-Nikodym theorem is valid).}
\begin{remark}
    {\color{blue} If X is a Banach space, then $X^*$ has the Radon-Nikodym property (RNP) if (and only if) every separable, linear
        subspace of X has a separable dual (Charles Stegall: The Radon-Nikodym property in Conjugate Banach Space. II)}.
\end{remark}
Thus for $p=1$ case, we can not use same argument as Theorem \ref{Poisson representation}. However there is a relative result. $L^1$ can be isometrically imbedded
in to $M$, the space of Borel measures with bounded variation, which is dual of continuous function with compact support space $C$. Thus we can use same argument to $M$ and $C$ and introduce Poisson-Stieltjes integral.\par
When we use Borel measures case in Theorem \ref{Poisson representation}, we get \emph{Poisson integral of positive measure for harmonic functions}.
\begin{equation}
    u(re^{i\theta})=\frac{1}{2\pi}\int_\pi^\pi{P_r(\theta-t)d\mu(t)} \label{Poisson integral 2}
\end{equation}
Here $d\mu(t)$ is the w*-limit of $u(r_ne^{it})d t$. The difference between \eqref{Poisson integral 1} and \eqref{Poisson integral 2} is that in \eqref{Poisson integral 1}, $u$ needs to be harmonic in a little larger
disk $D(0,R),R>1$, but in \eqref{Poisson integral 2} $u$ needs not. However, this is not hold for all  harmonic functions since we need constraints.\par

\subsection{Poisson integral indicates harmonic}\label{Poisson integral indicates harmonic}
Using Fourier series, we show that given a function $f\in L^p$, $1\leq p\leq \infty$ (or a complex Borel measure $\mu$), Poisson integral u=$P(f)$ (or u=$P(\mu)$) is real part of a holomorphic function. Thus it is harmonic (Theorem 1.11 (or 1.14) in book). And we give bound of norm of $u$ by norm of $f$ (or measure $\mu$):
\begin{itemize}
    \item $\int_{-\pi}^{\pi}\abs{u(re^{i t})}^p d t \leq \int_{-\pi}^{\pi}\abs{f(t)}^p d t~for~p<\infty$
    \item $\abs{u(z)}\leq \norm{f(t)}_\infty~for~p=\infty$
    \item $\int_{-\pi}^{\pi}\abs{u(re^{i t})} d t \leq \int_{-\pi}^{\pi}d\abs{\mu}(t)$
\end{itemize}
For $f\in L^p$, $p\leq\infty$ case, the equality holds when $r\to 1$ or take sup on right hand side. We prove this in section 3.
\subsection{Boundary behavior of Poisson integral (or norm convergence and pointwise convergence)}
\begin{definition}[Dirichlet problem]
    Given a continuous function $f$ on $\partial D$, we want to find a continuous function on $\bar{D}$, which is harmonic in $D$ and coincides with $f$ on $\partial D$.
\end{definition}
From section \ref{Poisson integral indicates harmonic}, we can see $u(re^{i t})=P(f)$ is harmonic function. Roughly speaking, if $u(e^{i t})=f$ (Actually the domain of u is D, so $u(e^{i t})$ is not defined),
the classical Dirichlet problem is solved. Thus we need to study the boundary behavior of Poisson integral. \emph{The key idea is approximate identity}. Using this idea, we prove the following theorem.\par
\begin{theorem}[Theorem 1.16 in book]\label{norm convergence}
    Let $\phi_\alpha$ be an approximate identity on the torus $T$. Then:
    \begin{enumerate}
        \item If $f\in L^p([-\pi,\pi])$ with $1\leq p<\infty$ and $f_\alpha$ stands for convolution
              \begin{equation*}
                  f_\alpha(\theta)=(f*\phi_\alpha)(\theta)=\frac{1}{2\pi}\int_{-\pi}^{\pi}{f(\theta-t)\phi_\alpha(t)d t}
              \end{equation*}
              it follows that $f_\alpha\to f$ in $L^p$, i.e.:
              \begin{equation*}
                  \int_{-\pi}^{\pi}{\abs{f_\alpha(t)-f(t)}^p d t}\to 0
              \end{equation*}
        \item If f is a continuous $2\pi$-periodic function, we have $f_\alpha\to f$ uniformly on $T$.
    \end{enumerate}
\end{theorem}
Theorem \ref{norm convergence} shows that for $f\in L^p$, $1\leq p<\infty$ or $f\in C$, Poisson integral converges to boundary in norm.
Following the proof of theorem \ref{norm convergence}, we can show another two case: $p=\infty$ and $f\in M$. The convergence becomes w* convergence (Corollary 1.19 in book).\par
Here comes another topic: What about pointwise convergence for $P(\mu)$ on boundary if $P(\mu)$ is Poisson-Stieltjes integral? We need a new concept: non-tangentially converge. We show that
$P(\mu)(z)\to F'(\theta_1)$ as $z\to e^{i\theta_1}$ N.T., where $F(\theta)=\int_0^\theta{d\mu(t)}$ and $F'(\theta_1)$ exists and finite (Theorem 1.20 in book).\par
You can omit the following remark if you choose not to check the proof in book.
\begin{remark}[Proof of Theorem 1.20]
    First we take $c>0$, since we need proof for any $c>0$. This $c$ decides the approach region.\\
    Then given $\epsilon>0$, we take $\delta$ small enough s.t. $\abs{F(t)}<\epsilon\abs{t}$ when $\abs{t}<\delta$. This $\delta$ can be taken since
    $F(0)=0$ and $F'(0)=0$.\\
    If we take $r$ large enough and $re^{i\theta}$ in the region. then $\abs{\theta}$ can be less than $\frac{\delta}{4}$.\\
    The estimation of $u(re^{i\theta})$ is (we use $F(t)=\int_{-\pi}^{\pi}{d\mu (t)}$ in the third line):
    \begin{align*}
        \abs{u(re^{i\theta})}= & \abs{\frac{1}{2\pi}\int_{\delta<\abs{t}\leq\pi}{P_r(\theta-t)d\mu(t)}+\frac{1}{2\pi}\int_{-\delta}^{\delta}{P_r(\theta-t)d\mu(t)}}                     \\
        \leq                   & \abs{\frac{1}{2\pi}\int_{\delta<\abs{t}\leq\pi}{P_r(\theta-t)d\mu(t)}}+\abs{\frac{1}{2\pi}\int_{-\delta}^{\delta}{P_r(\theta-t)d\mu(t)}}               \\
        \leq                   & (\sup_{{\color{red}\abs{t}>\delta}}{P_r(\theta-t)})\cdot\frac{1}{2\pi}\int_{\delta<\abs{t}\leq\pi}{d\abs{\mu}(t)}                                      \\
                               & +\abs{\left.(P_r(\theta-t)\cdot\frac{1}{2\pi}F(t))\right\rvert_{-\delta}^{\delta}
        +\int_{-\delta}^{\delta}{P_r'(\theta-t)\frac{1}{2\pi}F(t)d t}}                                                                                                                  \\
        \leq                   & (\sup_{{\color{red}\frac{3\delta}{4}<\abs{t}<\pi+\frac{\delta}{4}}}{P_r(t)})\cdot\frac{1}{2\pi}\int_{-\pi}^{\pi}{d\abs{\mu}(t)}                        \\
                               & +\abs{\left.(P_r(\theta-t)\cdot\frac{1}{2\pi}F(t))\right\rvert_{-\delta}^{\delta}}+\abs{\int_{-\delta}^{\delta}{(P_r'(\theta-t)\frac{1}{2\pi}F(t))dt}} \\
        \leq                   & (\sup_{{\color{red}\abs{t}>\frac{3\delta}{4}}}{P_r(t)})\cdot\frac{1}{2\pi}\int_{-\pi}^{\pi}{d\abs{\mu}(t)}                                             \\
                               & +(\sup_{{\color{red}\abs{t}>\frac{3\delta}{4}}}{P_r(t)})\cdot\frac{1}{2\pi}\int_{-\delta}^{\delta}{d\abs{\mu}(t)}+
        \abs{\frac{1}{2\pi}\int_{-\delta}^{\delta}{P_r'(\theta-t)F(t)dt}}
    \end{align*}
\end{remark}
Lebesgue-Stieltjes integral is related to bounded increasing function. The integral has decomposition $d\mu(t)=f(t)dt+d\sigma(t)$, $f\in L^1$.
$\int_0^\theta{d\sigma(t)}$ is a jump function. Thus $F'(\theta)=f(\theta)$ a.e.. Considering $P(f)$, $f\in L^p$, $1\leq p\leq\infty$, or even $P(\mu)$, the integral $P(f)$ or $P(\mu)$ is L-S integral. Thus N.T. convergence holds a.e..\par
By section \ref{harmonic function can be written as Poisson integral}, Harmonic function with constraints (bounded in some sense) can be written as Poisson integral $P(f)$ or $P(\mu)$ (Corollary 1.10 in book). Thus the theorem of Fatou
holds: \emph{Any function holomorphic and bounded in D has non-tangential boundary values a.e.}. \par
The difference between $p=1$ and $p>1$ is the starting point of the theory of Hardy spaces.
\subsection{Harmonic function in higher dimension}
A continuous function is harmonic in region $\Omega$ if and only if it satisfies mean value property. For mean value property to harmonic, we need an approximate
identity. \par
You can omit the following remark if you choose not to check the proof in book.
\begin{remark}[Proof of Theorem 1.22]
    In converse part, we choose region $\Omega_\epsilon$ is to make integral $\int_{\mathbb{R}^n}{u(x_0+r\sigma-y)\phi_\epsilon(y)dy}$ and $\int_{\mathbb{R}^n}{u(x_0-y)\phi_\epsilon(y)dy}$
    defined. In other words, if $x_0+r\sigma-y$ or $x_0-y$ out of $\Omega$, then $\phi_\epsilon(y)=0$.
\end{remark}
A consequence of mean value property is maximum principle. The proof can be given by topological technic: Assumed $u$ attains
maximum value m in $\Omega$. Let $A=\{x:u(x)=m\}$. Using mean value property, we show every $x$ is interior point of $A$. Thus $A$ is open.
However, $\Omega\setminus A=\{x:u(x)<m\}$ is open since $u$ is continuous. $\Omega$ is connected and $A$ is not empty. $\Omega\setminus A$ is empty. Thus u is constant.\par
By maximum principle and minimum principle, we can show $u$ is unique in $\Omega$ if $u$ is harmonic in $\Omega$ and $u$ is continuous on $\partial\Omega$.\par
We introduce the Poisson kernel $P(x,s)=\frac{1-\abs{x}^2}{\abs{x-s}^n}$ for Dirichlet problem in n dimension. The solution is
$\frac{1}{\abs{\Sigma_{n-1}}}\int_{\Sigma_{n-1}}{P(x,s)f(s)ds}$.\par
There is a weaken form of mean value property (or called discrete mean value property). $u$ is harmonic if mean value property is satisfied
only on a sequence of $r_n\to 0$ (Theorem 1.30 in book). \par
You can omit the following remark if you choose not to check the proof in book.
\begin{remark}[Proof of theorem 1.30]
    {\color{blue}The proof here is not well understand.} The set $K=\{u(x)-v(x)=m:x\in\overline{B(x_0,R)}\}$ is compact in $\overline{B(x_0,R)}$ since K is closed in compact set $\overline{B(x_0,R)}$. We can use
    finite open cover to prove K is also compact in $B(x_0,R)$. Let $f: K\to\mathbb{R}$, $f(x)=d(x,x_0)$, $f$ is continuous function on compact set
    K thus it attains maximum value of $f$. Let $x_1$ be a point of the maximum value. {\color{blue}I just can image that for sphere $\partial B(x_1,r)$, only half of
            the sphere is in K otherwise $f(x_1)$ is not the maximum value. Here $r$ need to be small enough to ensure $\partial B(x_1,r)\subset B(x_0,R)$}. But then $u(x_1)-v(x_1)<m$ if we use integral with $r_j<r$. Thus it is a contradiction. \par
    {\color{blue} I don't know
        how the sequence $(r_n)$ plays in this proof}.\par
\end{remark}
Then we go to the reflection principle and Liouville theorem.\par
Finally we go to Dirichlet problem on unbounded domain, in particular $\mathbb{R}_+^{n+1}$. This problem cannot have a unique solution but can have
a unique bounded solution. And we give Poisson kernel for $\mathbb{R}_+^{n+1}$ by Fourier transform method.

% \end{document}



