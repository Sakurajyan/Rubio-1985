\documentclass{report}
\usepackage{amsthm}
\usepackage{amsmath}
\usepackage{amssymb}
\usepackage{amsfonts}
\usepackage{xcolor}
\usepackage{mathrsfs}
\newtheorem{remark}{Remark}
\newtheorem{theorem}{Theorem}
\newtheorem{definition}[theorem]{Definition}
\newtheorem{lemma}[theorem]{Lemma}
\newtheorem{proposition}[theorem]{Proposition}
\newtheorem{corollary}[theorem]{Corollary}
\numberwithin{theorem}{subsection}
\numberwithin{remark}{subsection}
\newcommand{\norm}[1]{\lVert#1\rVert}
\newcommand{\abs}[1]{\left\lvert#1\right\rvert}
\newcommand{\absl}[1]{\lvert#1\rvert}
\renewcommand{\Re}{\operatorname{Re}}
\renewcommand{\Im}{\operatorname{Im}}
\newcommand{\sgn}{\operatorname{sgn}}
\begin{document}
\section{The Helson-Szego Theorem}
In this section, we no longer focus on Lebesgue measure. We will reexamine some results for boundedness of conjugate function on Borel measure.
The main result we give is the Helson-Szego theorem. The Helson-Szego theorem is a characterization of those positive (Borel) measure $\mu$ on $[-\pi,\pi]$ for which the conjugate function
operator is bounded in $L^2(\mu)$. More precisely, we answer the problem for what positive measure the following inequality holds:
\begin{equation*}
    \int_{-\pi}^{\pi}\absl{\tilde{f}(t)}^2 d\mu(t)\leq C\int_{-\pi}^{\pi}\abs{f(t)}^2 d\mu(t)
\end{equation*}
where $f(t)=\sum_{j=-n}^{n}{a_je^{ijt}}$ and $\tilde{f}(t)=-i\sum_{j=-n}^{n}{(\sgn j)a_je^{ijt}}$. $f(t)+i\tilde{f}(t)$ only left non negative frequency part.
We will call such measures Helson-Szego measures.
\subsection{Distance from constant function 1 to space $\mathscr{P}(0)$}
\begin{remark}[notes on proof of theorem 8.2 in book]
    To proof $1\in\overline{\mathscr{P}(0)}$, we can use Hahn-Banach theorem: if $E$ is locally convex and $F$ is subspace of $E$ and $x_0\in E$, then $x_0\in \bar{F}$ if and only if for
    any $f\in E^*$, $f\vert_F=0 \implies f(x_0)=0$\par
    $\mu$ and $\nu$ are mutually singular if there are disjoint subsets $A$ and $B$ in $\mathscr{M}$ s.t. $\nu(E)=\nu(A\cap E)$ and
    $\mu(E)=\mu(B\cap E)$ for all $E\in\mathscr{M}$. $\nu$ is absolutely continuous w.r.t. $\mu$ if $\nu(E)=0$ whenever $E\in\mathscr{M}$ and $\mu(E)=0$ (section 4.2 in Chapter 6 of Stein’s Real Analysis).\par
    Thus if $\mu$ and $\nu$ are mutually singular and $\nu$ is absolutely continuous w.r.t. $\mu$, then for all $E\in\mathscr{M}$, $\mu(A\cap E)=\mu(A\cap B\cap E)=0$ and hence $\nu(E)=\nu(A\cap E)=0$. Therefore $\nu$ is identically zero.
\end{remark}
The following lemma is useful in specifying Helson-Szego measure. It states there is a holomorphic function continuous on boundary can separate closed Lebesgue measure 0 set $E$ and set $\bar{D}\setminus E$. Using this lemma, we can narrow our candidates non negative measure to absolutely continuous measure. All Helson-Szego measure $d\mu$ can be written as $d\mu=w(t)dt$.\par
\begin{lemma}[lemma 8.3 in book]\label{separate zero measure}
    Let $E$ be a closed subset of $T$ having Lebesgue measure 0. Then, there exists a function $F\in\mathscr{A}$ (that is: $F$ is holomorphic in $D$ and
    continuous on $\bar{D}$) such that $F(z)=1$ for every $z\in E$ and $\abs{F(z)}<1$ for every $z\in\bar{D}\setminus E$.
\end{lemma}
\begin{remark}[notes on proof of theorem 8.4 in book]
    {\color{blue}In theorem 8.4 in book we assumes $w\geq 0$ but in proof this condition is never used}.
    Suppose $\mu$ is a Borel measure and is finite on all finite radius balls. Then for any Borel set $E$ and any $\epsilon>0$, there are an open
    set $O$ and a closed set $F$ such that $E\subset O$ and $\mu(O-E)\leq\epsilon$ while $F\subset E$ and $\mu(E-F)\leq\epsilon$ (proposition 1.3 in Chapter 6 of Stein’s Real Analysis).\par
    By above statement, inequality (8.6) in book can be hold.\par
\end{remark}
The following theorem gives a precise formula for the distance we are looking for:
\begin{theorem}[theorem 8.7 in book]\label{distance from P(0) to 1}
    For $w\geq 0$, $w\in L^1$:
    \begin{equation*}
        \inf_{P\in\mathscr{P}(0)}\frac{1}{2\pi}\int_{-\pi}^{\pi}\abs{1-P(e^{it})}^p w(t)d t=\exp(\frac{1}{2\pi}\int_{-\pi}^{\pi}\log w(t)d t),1\leq p<\infty
    \end{equation*}
\end{theorem}
\subsection{The Helson-Szego theorem}

By theorem \ref{distance from P(0) to 1}, if $\log w(t)=-\infty$,  there is a sequence of $(P_j)\subset\mathscr{P}(0)$, s.t. $\int_{-\pi}^{\pi}\abs{1-P_j(e^{it})}^2 w(t)d t\to 0$. Since the conjugate function of $1-P_j(e^{it})$ is $iP_j(e^{it})$ and $\mu$ is
Helson-Szego measure, we have $\int_{-\pi}^{\pi}\abs{P_j(e^{it})}^2 w(t)d t\to 0$. By $1\leq\abs{P_j(e^{it})}^2+\abs{1-P_j(e^{it})}^2$, we show $\int_{-\pi}^{\pi}1 w(t)d t= 0$. We conclude $\log w(t)=-\infty$ implies $\mu$ is a trivial measure. Thus we can only
focus on Helson-Szego measure $\mu$ which $d\mu=w(t)dt$, $w(t)\in L^1$ and $\log w(t)\in L^1$.\par
Given $f(t)=\sum_j{a_je^{ijt}}$, operator $A$ sends $f(t)$ to $Af(t)=\sum_{j>0}{a_je^{ijt}}$   . By easy calculation we have following identities:
\begin{equation*}
    \tilde{f}=-i(A f-\overline{A\bar{f}})~and~A f=\frac{-\tilde{\tilde{f}}+i\tilde{f}}{2}
\end{equation*}
\begin{remark}[proof of lemma 8.10]

    This proof is from P165 in  Paul Koosis' \emph{Introduction to Hp spaces}.\par
    By $\tilde{f}=-i(Af-\overline{A\bar{f}})$, if $\norm{Af}_w\leq C\norm{f}_w$, then
    \begin{equation*}
        \norm{\tilde{f}}_w\leq \norm{Af}_w+\norm{\overline{A\bar{f}}}_w=\norm{Af}_w+\norm{{A\bar{f}}}_w\leq C\norm{f}_w+C\norm{{\bar{f}}}_w\leq 2C\norm{f}_w
    \end{equation*}
    By $Af=\frac{-\tilde{\tilde{f}}+i\tilde{f}}{2}$, if $\norm{\tilde{f}}_w\leq C\norm{f}_w$, then
    \begin{equation*}
        \norm{Af}_w\leq \frac{1}{2}\norm{\tilde{\tilde{f}}}_w+\frac{1}{2}\norm{\tilde{f}}_w\leq \frac{C}{2}\norm{{\tilde{f}}}_w+\frac{C}{2}\norm{{f}}_w\leq \frac{C^2}{2}\norm{f}_w+\frac{C}{2}\norm{f}_w\leq \frac{C^2+C}{2}\norm{{f}}_w
    \end{equation*}
    Thus conjugate operator bounded in $L^2$ is equivalent to operator $A$ bounded in $L^2$.
\end{remark}
We now talk about how Helson-Szego theorem reached. Every lemmas or theorems contain different ideas in the path to Helson-Szego theorem.
Lemma 8.10 in book shows we can study boundedness of operator $A$ instead of boundedness of conjugate operator. In theorem 8.11 in book, if we write $f(t)=P(e^{it})+e^{it}\overline{Q(e^{it})}$ for some
$P,Q\in\mathscr{P}(0)$, we can study the boundedness in space $\mathscr{P}(0)$ but we need an additional restriction on some constants.
Later we state theorem 8.12 in book, we further shows the restriction on the constant in theorem 8.11 in book becomes the restriction on the norm in quotient space $(H^1(0))^*=L^\infty/ H^\infty$.
The norm on quotient space is naturally the distance from represent element to space $H^\infty$.\par
Now we state the Helson-Szego theorem:
\begin{theorem}[Helson-Szego Theorem, theorem 8.14 in book]
    $w$ is a Helson-Szego weight if and only if $w(t)=e^{u(t)+\tilde{v}(t)}$ with $u$ and $v$ real, bounded and $\norm{v}_\infty<\frac{\pi}{2}$
\end{theorem}
\begin{remark}[notes on proof of Helson-Szego theorem]
    I also refer another proof of Helson-Szego theorem in P167 in Paul Koosis' \emph{Introduction to Hp spaces}.\par
    Since $u$ is bounded, we have $0<C_1<e^u<C_2<\infty$. If $w(t)=e^{\tilde{v}(t)}$ is a Helson-Szego measure, then:
    \begin{equation*}
        \int_{-\pi}^{\pi}\absl{\tilde{f}(t)}^2e^{u+\tilde{v}}d t\leq \int_{-\pi}^{\pi}\absl{\tilde{f}(t)}^2C_2e^{\tilde{v}}d t\leq C\int_{-\pi}^{\pi}\absl{{f}(t)}^2C_2e^{\tilde{v}}d t\leq C\int_{-\pi}^{\pi}\absl{{f}(t)}^2\frac{C_2}{C_1}e^{u+\tilde{v}}d t
    \end{equation*}
    Thus we just need to consider the case $w(t)=e^{\tilde{v}(t)}$.\par
    Recall conjugate function is real. Suppose $\Phi(z)=\exp{(\frac{1}{2\pi}\int_{-\pi}^{\pi}\frac{e^{it}+z}{e^{it}-z}\frac{1}{2}\tilde{v}(t)dt)}$.
    We have  $\abs{\Phi(z)}^2=\exp{(P(\tilde{v}))}$ and $\Phi(z)^2=\exp{(\frac{1}{2\pi}\int_{-\pi}^{\pi}\frac{e^{it}+z}{e^{it}-z}\tilde{v}(t)dt)}$. Now we compute the angle $\frac{\Phi^2}{\abs{\Phi}^2}$:
    \begin{align*}
        \frac{\Phi(z)^2}{\abs{\Phi(z)}^2} & =\exp{(\frac{1}{2\pi}\int_{-\pi}^{\pi}(\frac{e^{it}+z}{e^{it}-z}-P_r(\theta-t))\tilde{v}(t)d t)} \\
                                          & =\exp{(\frac{1}{2\pi}\int_{-\pi}^{\pi}iQ_r(\theta-t)\tilde{v}(t)d t)}                            \\
                                          & =\exp{(i\tilde{\tilde{v}}(re^{i\theta}))}                                                        \\
    \end{align*}
    If $v(t)=\sum_{j}{a_je^{ijt}}$, then $\tilde{v}(t)=-i\sum_{j}{(\sgn j)a_je^{ijt}}$ and $\tilde{\tilde{v}}(t)=-\sum_{j\neq 0}{a_je^{ijt}}=-v(t)+a_0=-v(t)+v(0)$. Thus $\frac{\Phi^2}{\abs{\Phi}^2}=\exp{(-iP(v)+iv(0))}$.\par
    Now we have
    \begin{align*}
        \Phi(z)^2 & =\abs{\Phi(z)}^2\frac{\Phi(z)^2}{\abs{\Phi(z)}^2} \\
                  & =\exp{(P(\tilde{v}))}\exp{(-iP(v)+iv(0))}         \\
                  & =\exp{(P(\tilde{v}-iv))}\exp{iv(0))}
    \end{align*}
    Thus we can see the real number $\tau$ in book is $v(0)$. $\tau$ is irrelevant since $\inf_{g\in H^\infty}{\norm{e^{-i\tau}e^{iv(t)}-g}_\infty}=\inf_{g\in H^\infty}{\norm{e^{iv(t)}-g}_\infty}$. Notice that constants are in $H^\infty$.
    Thus $\abs{e^{iv(t)}-sin\epsilon}<1$ implies $d(e^{iv},H^\infty)<1$.\par
    The inequality $\abs{e^{i\phi(t)}-H(e^{it})}\leq\alpha$ in book should be $\abs{e^{i\phi(t)}-H(e^{it})}\leq\alpha_1$ for some $\alpha_1$ with $\inf_{g\in H^\infty}{\norm{e^{iv(t)}-g}_\infty}=\alpha<\alpha_1<1$.\par
    An outer function $F$ has representation:
    \begin{equation*}
        F(z)=C\exp{(\frac{1}{2\pi}\int_{-\pi}^{\pi}{\frac{e^{it}+z}{e^{it}-z}w(t)d t})}
    \end{equation*}
    where $C=I_F(z)$ is constant. Decompose $\frac{e^{it}+z}{e^{it}-z}=P_r(\theta-t)+iQ_r(\theta-t)$, we have:
    \begin{equation*}
        F(z)=C\exp{(\frac{1}{2\pi}\int_{-\pi}^{\pi}{(P_r(\theta-t)+iQ_r(\theta-t))w(t)d t})}=C\exp{(u(re^{i\theta})+iv(re^{i\theta}))}
    \end{equation*}
    where $u$ is a real harmonic function and $v$ its conjugate. Let $r\to 1$ and notice $\abs{C}\to 1$. Thus we have $F(e^{it})=e^{i\tau}e^{u+iv}$.\par
    We conclude that an outer function has representation $e^{i\tau}e^{u+iv}$ where $\tau$ is a real number. $u$ is a real harmonic function and $v$ its conjugate.
\end{remark}
\end{document}

