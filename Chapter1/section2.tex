% \documentclass{report}
% \usepackage{amsthm}
% \usepackage{amsmath}
% \usepackage{amsfonts}
% \usepackage{xcolor}
% \usepackage{mathrsfs}
% \newtheorem*{remark}{Remark}
% \newtheorem*{definition}{Definition}
% \newtheorem{theorem}{Theorem}
% \newtheorem{proposition}{Proposition}
% \numberwithin{theorem}{subsection}
% \newcommand{\norm}[1]{\left\lVert#1\right\rVert}
% \newcommand{\abs}[1]{\left\lvert#1\right\rvert}
% \begin{document}
\section{Subharmonic Functions}
This section is about a new concept: subharmonic function. Subharmonic function can be considered as a generalization of harmonic
function, as it preserves some important property of harmonic function such as maximum principle. On the other hand, we will see
why we call it "sub" harmonic: subharmonic function can be controlled by harmonic function. Also, by some operations like
composition and taking absolute value, subharmonic function can still be subharmonic, but harmonic function can not. Finally we will begin our study of zeroes of holomorphic function.\par

Now we give the definition of subharmonic function.
\begin{definition}
    A subharmonic function on an open set $\Omega\subset\mathbb{R}^n$ is a function v defined on $\Omega$, with values $-\infty\leq v(x)<\infty$
    and satisfying the following two conditions:
    \begin{enumerate}
        \item v is upper semicontinuous in $\Omega$.
        \item For every $x_0\in\Omega$, there is a ball $B(x_0,r(x_0))\subset\Omega$, $r(x_0)>0$, such that for every r with $0<r<r(x_0)$
              \begin{equation}
                  v(x_0)\leq\frac{1}{\abs{\Sigma_{n-1}}}\int_{\Sigma_{n-1}}{v(x_0+r\sigma) d\sigma}\label{local submean property}
              \end{equation}
    \end{enumerate}
\end{definition}
\subsection{Upper semicontinuous}
There is two equivalence definition of $v$ being upper semicontinuous in $\Omega$:
\begin{enumerate}
    \item For every $t\in\mathbb{R}$, the set $\{x\in\Omega: v(x)<t\}$ is open.
    \item For every $x_0\in\Omega$:
          \begin{equation}
              \limsup_{x\to x_0~in~\Omega}{v(x)\leq v(x_0)}\label{upper semicontinuous}
          \end{equation}
          This is equivalent to that for every $y>f(x_{0})$, there exists a neighborhood $U$ of $x_{0}$ such that $f(x)<y$ for all $x \in U$.
\end{enumerate}
\begin{remark}[Proof of equivalence]\ \\
    We prove by contradiction, Suppose that $\limsup{v(x)>v(x_0)}$, we can find $x_k$, $v(x_0)<v(x_k)<\limsup{v(x)}$. Since $v^{-1}([-\infty,v(x_k))$ is open and $v(x_0)<v(x_k)$,
    $v(x_0)\in v^{-1}([-\infty,v(x_k))$. Thus there is a neighborhood $U\in\mathscr{N}(x_0)$, $U\subset v^{-1}([-\infty,v(x_k))$. Now we can find another $x_n\in U$
    s.t. $v(x_k)<v(x_n)<\limsup{v(x)}$. $v(x_n)>v(x_k)$ means $v(x_n)\notin v^{-1}([-\infty,v(x_k))$, contradicts to $v(x_n)\in U\subset v^{-1}([-\infty,v(x_k))$.\par
    We prove the converse by contradiction. Suppose there is a number $t_0\in\mathbb{R}$, $v^{-1}([-\infty,t_0))$ is not open. So there is
    $x_0\in v^{-1}([-\infty,t_0))$, such that $\forall U_k\in\mathscr{N}(x_0)$, there is $x_k\in U_k$, $x_k\notin v^{-1}([-\infty,t_0))$, which is equivalent to $v(x_k)\geq t$. This contradicts to $\limsup{v(x_k)}\leq v(x_0)<t_0$.
\end{remark}
If $v$ is subharmonic, inequality \eqref{local submean property} implies another direction of inequality \eqref{upper semicontinuous}.
Thus we actually have equality in \eqref{upper semicontinuous}.\par
An important and frequently used tool is following characterization of upper simicontinuity.
\begin{proposition}\label{characterization of semicontinuous}
    $v$ is upper semicontinuous in $\Omega$ if and only if for every compact $K\subset\Omega$, $v$ is the limit over $K$ of a
    decreasing sequence of continuous function.
\end{proposition}
This proposition is important tool in proof of some following theorems.
\begin{remark}[Notes on proof of proposition \ref{characterization of semicontinuous}]
    The converse part, by using partition of the unity, we construct a sequence of decreasing function $(u_k)$. We need to prove $v$
    is the limit of $(u_k)$. \par
    For any $x_0\in K$, there is a sequence of balls $(B(x_{n,i},\epsilon_n))$, $\epsilon_n\to 0$ s.t. $x_0\in B(x_{n,i},\epsilon_n)$ for all $n$.
    For each $n$, $B(x_{n,i},\epsilon_n)$ is in finite open cover $(B(x_{n,i},\epsilon_n))_i$ of $K$. Since $m_{n,i}=\sup_{B(x_{n,i},\epsilon_n)}{v}$,
    $u_n(x_0)\geq v(x_0)$ for all $n$. By definition of upper semicontinuous, for any $y>v(x_0)$, there is a neighborhood $U\in \mathscr{N}(x_0)$,
    $v(x)<y$ for all $x\in U$. Let $B(x_{n,i},\epsilon_n)\subset U$, $m_{n,i}<y$. Thus $u_n(x_0)<y$ for all large enough $n$. Since $y$ is any number larger than $v(x_0)$, $\limsup{u(x)}\leq v(x_0)$.   This shows $u(x_0)=v(x_0)$.
\end{remark}

\subsection{Property of subharmonic function}
First, subharmonic function satisfies maximum principle.
\begin{remark}[Notes on proof of maximum principle]
    Like proof of maximum principle for harmonic function, but we need
    to take care of semicontinuous. Assume $v(x_0)=M$, the maximum value. Choose $r$ to satisfy inequality \eqref{local submean property}.
    If for some $x\in \partial B(x_0,r)$, $v(x)=m<M$, by semicontinuous, $\limsup{v(x_k)}\leq v(x)<m+\epsilon<M$. Thus there is a
    neighborhood $U\in\mathscr{N}(x)$, $\sup_{x_k\in U}{v(x_k)}<M$. Then $\frac{1}{\abs{\Sigma_{n-1}}}\int_{\Sigma_{n-1}}{v(x_0+r\sigma) d\sigma}<M=v(x_0)$.
    This contradicts to inequality \eqref{local submean property}. Then the following is same as proof for maximum principle for harmonic function.
\end{remark}
The best reason why we use name 'subharmonic' is following: $v$ is subharmonic function if and only if when $v$ less or equal to a harmonic function $u$
on boundary of region, $v\leq u$ in entire region. We remind the reader that proposition \ref{characterization of semicontinuous} appears as an important
step in proof.\par
There are two examples of using proposition \ref{characterization of semicontinuous} to detail with subharmonic function $v$. One is if $v$
is not identically equal to $-\infty$, then
\begin{equation*}
    \frac{1}{\abs{\Sigma_{n-1}}}\int_{\Sigma_{n-1}}{v(x_0+r\sigma) d\sigma}>-\infty
\end{equation*}
for every $\overline{B(x_0,r)}\subset\Omega$.
In proof of this statement we also use Poisson representation of harmonic function and little topological trick. Another example
is
\begin{equation}\label{equation m(r)}
    m(r)=\frac{1}{\abs{\Sigma_{n-1}}}\int_{\Sigma_{n-1}}{v(r\sigma) d\sigma}
\end{equation}
is an increasing function.\par
There is another necessary and sufficient condition for $v$ to be harmonic using Laplace operator. It says $v$ is subharmonic if and
only if $\Delta v\geq 0$.
\begin{remark}[Proof of proposition 2.10 in book]
    {\color{blue}Author says we need to show that $v(x_0)\leq\frac{1}{\abs{\Sigma_{n-1}}}\int_{\Sigma_{n-1}}{v(x_0+r\sigma) d\sigma}$. But I
        think this is obvious since we consider $x_0$ which $v(x_0)=0$ and $v(x)\geq 0$ on $\Omega$. And this inequality is not used in the following part of proof}.
\end{remark}

\subsection{Estimation for zeroes of holomorphic function}
We first state that if $v$ is subharmonic, $\phi$ is increasing and convex function. Then $\phi\circ v$ is also subharmonic. This is useful when we need to connect holomorphic function with subharmonic function.\par

Now here comes our first theorem about zero points of holomorphic: Jensen's formula.
\begin{theorem}[Jensen's formula]
    Let F be holomorphic in $D(0,R)$ and suppose that $F(0)\neq 0$. Let $0<r<R$ and call $z_1,z_2,\cdots,z_n$ the zeroes of $F$
    in $\overline{D(0,r)}$ listed according to their multiplicities. Then:
    \begin{equation}
        \log{\abs{F(0)}}+\sum_{j=1}^{n}{\log{\frac{r}{\abs{z_j}}}}=\frac{1}{\pi}\int_{-\pi}^{\pi}{\log{\abs{F(re^{it})}}d t}.
    \end{equation}
\end{theorem}
The proof in book need a lemma: $\int_{-\pi}^{\pi}{\log{\abs{1-e^{it}}}d t}=0$. You can also refer section 1 in Chapter 6 of Stein's \emph{Complex Analysis}. The proof there is very different to the one in this book.\par
\begin{remark}[Proof of lemma (Lemma 2.12 in book)]
    {\color{blue}There is an inequality: for $\abs{t}<\frac{\pi}{3}$, $\log{\frac{1}{\abs{\sin{t}}}}\leq \frac{C_\alpha}{\abs{t}^\alpha}$. I can prove it
        using elementary calculus, but I think it is an easy observation.}
\end{remark}
To continue our explorer of zeroes of holomorphic function, we show some connection between holomorphic function and subharmonic
function. More precisely, If $F$ is holomorphic, not identically 0, then $\log{\abs{F(z)}}$, $\log{^+\abs{F(z)}}=\max{(\log{\abs{F(z)}},0)}$
and $\abs{F(z)}^a$ for any $0<a<\infty$, are all subharmonic. Then we give definition of Hardy space on $D$. We define for
$f\in H(D)$ ($F$ is holomorphic in $D$) :
\begin{itemize}
    \item $m_0(F,r)=\exp{(\frac{1}{2\pi}\int_{-\pi}^\pi{\log{^+\abs{F(re^{it})}}d t})}$
    \item $m_p(F,r)=(\frac{1}{2\pi}\int_{-\pi}^\pi{\abs{F(re^{it})}^p d t})^\frac{1}{p}$
    \item $m_\infty(F,r)=\sup_t{\abs{F(re^{it})}}$
\end{itemize}
This function is an increasing function of $r$ in $[0,1)$ (Hardy convex theorem),
see equation \eqref{equation m(r)} for case $0\leq p<\infty$. $m_\infty(F,r)$ is also an increasing function but it uses a different method {\color{blue}(Hadamard three-circle theorem)}. \par
Now we define Hardy space $H^p$:
\begin{definition}
    For $0<p\leq\infty$, we define $H^p(D)$:
    \begin{equation*}
        H^p(D)=\{F\in H(D):\norm{F}_{H^p}=\sup_{0\leq r<1}{m_p(F,r)<\infty}\}
    \end{equation*}
    For $p=0$, we have the Nevanlinna class $N$, defined by:
    \begin{equation*}
        N=\{F\in H(D):\sup_{0\leq r<1}{m_0(F,r)<\infty}\}
    \end{equation*}
    If $0<p<q<\infty$, we have $H^\infty\subset H^q\subset H^p\subset N$
\end{definition}
\begin{remark}
    The first two inclusions are as the same as the inclusion for $L^p$, the last inclusion is by:
    \begin{equation*}
        (m_0(r))^p=\exp{(p\int_{-\pi}^{\pi}{\log{^+\abs{F}}\frac{dt}{2\pi}})}\leq\int_{-\pi}^{\pi}{\exp{(p\log{^+\abs{F}})\frac{dt}{2\pi}}}
    \end{equation*}
    Notice that:
    \begin{equation*}
        \int_{-\pi}^{\pi}{\exp{(p\log{^+\abs{F}})\frac{dt}{2\pi}}}=\int_{\substack{t\in [-\pi,\pi]\\ \abs{F}> 1}}{\abs{F}^p \frac{dt}{2\pi}}+\int_{\substack{t\in [-\pi,\pi]\\ \abs{F}\leq 1}}{1 \frac{dt}{2\pi}}
    \end{equation*}
    Thus:
    \begin{equation*}
        (m_0(r))^p\leq\int{\abs{F}^p\frac{dt}{2\pi}}+1
    \end{equation*}
\end{remark}
There left three theorems in this section. I interpret it shortly and informally. First one is for $F\in N$, the zeroes $(z_j)$
of F cannot be too far from the boundary, or $\sum_j{(1-\abs{z_j})<\infty}$. The second one is that if $\sum_j{(1-\abs{z_j})<\infty}$ holds, the "Blaschke product"
\begin{equation*}
    B(z)=z^k\prod_{j=1}^\infty{\frac{z_j-z}{1-z\bar{z_j}}\frac{\abs{z_j}}{z_j}}
\end{equation*}
converges uniformly on each compact subset to a function $H^\infty$ and they have the same zeroes.\par
\begin{remark}
    If $f$ is holomorphic in an open disc that vanishes on a sequence of distinct points with a limit point in the disc.
    Then f is identically 0 (Theorem 4.8 in chapter 2. Stein's Complex Analysis). However in Blaschke product case, there can
    be infinitely zeroes, since it can have limit points on boundary. Thus $B(z)$ can be not identically 0. However, for any $r<1$,
    $B(z)$ can only have finite zeroes in $\overline{D(0,r)}$.
\end{remark}
Here is a convention. If for some function $F$ in $D$, the non-tangential boundary value of F is known to exist at $e^{it}$, we shall denote it by
$F(e^{it})$. \par
The third theorem is the Blaschke
product has properties: $\abs{B(e^{it})}=1$ for a.e. $t$ and
\begin{equation*}
    \lim_{r\to 1}{\frac{1}{2\pi}\int_{-\pi}^\pi{\log{\abs{B(re^{it})}}d t}}=0.
\end{equation*}\par
If $F\in H^p$ with $p\geq 1$, which means $F$ is holomorphic $F$ and can be write as Poisson (or Poisson-Stieltjes) integral. By Fatou Theorem, we know that $F(e^{it})$ exists a.e..
In the next section we shall extend this result to any $p>0$. The above three theorems play important roles in proving extension.\par
\begin{remark}[Notes on proof of three theorems (theorem 2.19, 2.21 and 2.22)]
    In proof of theorem 2.19, we assume $F(0)=0$, since we can use function $\frac{F(z)}{z^k}$ if $F(z)$ has zero of order k in $z=0$.
    And this modification does not affect the sum $\sum_j{(1-\abs{z_j})}$.\par
    {\color{blue} The step
        \begin{equation*}
            \sum_1^n{\log{\frac{1}{\abs{z_j}}}}\leq M-n\log{r}-\log{\abs{F(0)}}
        \end{equation*}
        to
        \begin{equation*}
            \sum_1^\infty{\log{\frac{1}{\abs{z_j}}}}\leq M-\log{\abs{F(0)}}
        \end{equation*}
        is not clear. I think we can not first let $r\to 1$ then $n\to \infty$. We can not control taking limit for which one first.
    }\par
    In proof of theorem 2.21, the final step is:
    \begin{align*}
        \abs{1-\frac{z_j-z}{1-z\bar{z_j}}\frac{\abs{z_j}}{z_j}} & =\abs{1-\frac{z_j\abs{z_j}-z\abs{z_j}}{z_j-z\abs{z_j}^2}}
        =\abs{\frac{z_j-z\abs{z_j}^2-z_j\abs{z_j}+z\abs{z_j}}{z_j-z\abs{z_j}^2}}                                                                                                             \\
                                                                & =(1-\abs{z_j})\abs{\frac{z_j+z\abs{z_j}}{z_j-z\abs{z_j}^2}}=(1-\abs{z_j})\abs{z_j}\abs{\frac{e^{it}+z}{e^{it}-z\abs{z_j}}} \\
                                                                & =(1-\abs{z_j})\abs{z_j}\abs{\frac{z'+1}{1-z'\abs{z_j}}}
    \end{align*}
    where $z'=z\cdot e^{-it}$. Since $\abs{z_j}<1$, $\abs{z'}=\abs{z}\leq r$, we have $\abs{z'+1}\leq\abs{z'}+1\leq r+1$, $\abs{1-z'\abs{z_j}}\geq 1-\abs{z'\abs{z_j}}=1-\abs{z'}\abs{z_j}\geq 1-r$. Thus
    \begin{equation*}
        \abs{1-\frac{z_j-z}{1-z\bar{z_j}}\frac{\abs{z_j}}{z_j}}\leq(1-\abs{z_j})\frac{1+r}{1-r}
    \end{equation*}\par
    In proof of theorem 2.22, we know if $z\bar{w}\neq 1$, then Blaschke factors:
    \begin{equation*}
        \abs{\frac{w-z}{1-\bar{w}z}}=1~if~\abs{z}=1~or~\abs{w}=1
    \end{equation*}
    {\color{blue}Since in $B_n(z)$ $\abs{e^{it}}=1$, I think $B_n(e^{it})=1$ everywhere, not a.e.}.\par
    $\abs{B_n(re^{it})}\to 1$ uniformly as $r\to 1$, since $B_n$ is holomorphic in a neighborhood of $\bar{D}$. This is easy
    if we choose $D(0,1+\epsilon)$, s.t. $z\bar{z_j}\neq 1$ in $D(0,1+\epsilon)$.
\end{remark}
% \end{document}
