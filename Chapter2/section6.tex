\documentclass{report}
\usepackage{amsthm}
\usepackage{amsmath}
\usepackage{amssymb}
\usepackage{amsfonts}
\usepackage{xcolor}
\usepackage{mathrsfs}
\usepackage[framemethod=tikz]{mdframed}


\theoremstyle{definition}
\newmdtheoremenv[
  hidealllines=true,
  leftline=true,
  innerleftmargin=10pt,
  innerrightmargin=10pt,
  innertopmargin=0pt,
]{remark}{Remark}
\newtheorem{note}{Note}

\theoremstyle{definition}
\newmdtheoremenv[
    hidealllines=true,
    innerleftmargin=0pt,
    innerrightmargin=0pt,
    fontcolor=red
]{errata}{Errata}
% \newtheorem{remark}{Remark}

\theoremstyle{plain}
\newtheorem{theorem}{Theorem}
\newtheorem{definition}[theorem]{Definition}
\newtheorem{lemma}[theorem]{Lemma}
\newtheorem{proposition}[theorem]{Proposition}
\newtheorem{corollary}[theorem]{Corollary}
\numberwithin{theorem}{subsection}
\numberwithin{remark}{subsection}
\newcommand{\norm}[1]{\lVert#1\rVert}
\newcommand{\abs}[1]{\left\lvert#1\right\rvert}
\newcommand{\absl}[1]{\lvert#1\rvert}
\renewcommand{\Re}{\operatorname{Re}}
\renewcommand{\Im}{\operatorname{Im}}
\newcommand{\sgn}{\operatorname{sgn}}
\newcommand{\sign}{\operatorname{sign}}
\newcommand{\BMO}{\operatorname{BMO}}
\newcommand{\argmax}{\operatorname{argmax}}
\newcommand{\supp}{\operatorname{supp}}

\begin{document}
\section{Multipliers}
In this section, we keep talk about the convergence property and norm estimate of convolution operator. In last section, we know
some convolution operator can not be defined as usual Lebesgue integral but a limit process called singular integral. The stimulation of
studying the convolution operator is that the convolution operator is equivalent to translation invariant operator.\par
Suppose $Tg=\int f(x-y)g(y)dy$ is a convolution operator. Then by Fourier transform, we have $(Tf)^\wedge=\hat{f}\hat{g}$. The  concept of multipliers give us another approach to study the convolution operator $T$:
\begin{definition}
    Let $1\leq p<\infty$. Given $m\in L^\infty$, we say that $m$ is a (Fourier) multiplier for $L^p$ if the operator
    $T_m$, initially defined in $L^2$ by the relation:
    \begin{equation*}
        (T_mf)^\wedge(\xi)=m(\xi)\hat{f}(\xi)
    \end{equation*}
    satisfies the inequality
    \begin{equation*}
        \norm{T_mf}_p\leq C\norm{f}_p  \quad (f\in L^2\cap L^p)
    \end{equation*}
\end{definition}

\begin{remark}
    {\color{blue} Given a sequence of kernels $(k_N)_N$ in $L^1$, if $T_Nf=k_N*f$ are uniformly bounded from $\mathscr{S}$ to $L^p$,
        it is easy to see $(T_N)_N$ is a Cauchy sequence of in bounded linear functional space $\mathscr{B}(\mathscr{S},L^p)$ ($\mathscr{B}(\mathscr{S},L^p)$ is complete by
        assigning $L^p$ norm on $\mathscr{S}$ and completeness of $L^p$), then the limit of $T_N$ exists.
        Indeed, under hypothesis, proposition 5.5 in book gives $T_Nf$ is a singular integral for $f\in \mathscr{S}(\mathbb{R}^n)$ and $T_N$ is uniformly bounded.
        Since $\mathscr{S}(\mathbb{R}^n)$ is dense in $L^p$, $T$ can be uniquely extends to $L^p$}.\par
    Notice that the limit process agrees the definition of singular integral.
\end{remark}

\subsection{Hormander-Mihlin multiplier theorem}
We know every $m\in L^\infty$ is a multiplier for $L^2$ by Plancherel's theorem. The Hormander-Mihlin multiplier theorem gives a
sufficient condition for $m\in L^\infty$ is a multiplier for $L^p$.
\begin{theorem}[Hormander-Mihlin multiplier theorem]
    Let $a=[\frac{n}{2}]+1$ be the first integer greater than $\frac{n}{2}$. If $m\in L^\infty(\mathbb{R}^n)$ is of class $C^a$ outside the origin
    and satisfies:
    \begin{equation}\label{hypothesis of H-M multiplier theorem}
        (R^{-n}\int_{R<\abs{x}<2R}\abs{D^\alpha m(\xi)}^2d\xi)^{\frac{1}{2}}\leq CR^{-\abs{\alpha}}\quad (0<R<\infty)
    \end{equation}
    for every multi-index $\alpha$ such that $\abs{\alpha}\leq a$, then $m$ is a multiplier for $L^{p}$, $1<p<\infty$.
\end{theorem}
This theorem is an improvement of Mihlin multiplier theorem. Mihlin multiplier theorem supposes a
stronger condition:
\begin{equation}\label{ieq: Mihlin condition}
    \abs{D^\alpha m(\xi)}\leq CR^{-\abs{\alpha}}\quad (\abs{\alpha}\leq a)
\end{equation}
In other words, Hormander weaken the decreasing speed as $\alpha$ increasing. The decreasing of  uniform boundedness of $\abs{D^\alpha m(\xi)}$ is weakened to the decreasing of $L^{2}$ norm of $\abs{D^\alpha m(\xi)}$.\par
Inequality \eqref{ieq: Mihlin condition} is satisfied by every function $m(\xi)$ of class $C^{a}$ outside the origin of degree $ib$.
Indeed by chain rule: $\frac{\partial}{\partial x_i}(f_i(tx))=f_i(tx)t$. And $\frac{\partial}{\partial x_i}(t^pf_i(x))=t^pf_i(x)\leq C\abs{\xi}^{-1}$. We have
$f_i(tx)t=t^pf_i(x)$. Thus $\abs{m_i(\xi)}=\abs{\abs{\xi}^{ib-1}m_i(x')}=\frac{\abs{m_i(x')}}{\abs{\xi}}$ with $\abs{x'}=1$.
By induction, $\abs{D^\alpha m(\xi)}\leq \frac{\abs{D^\alpha m(x')}}{\xi^{-\alpha}}$. Let
\begin{equation*}
    C=\max_{\abs{\alpha}\leq a}\sup_{\abs{x'}=1}\abs{D^\alpha m(x')}
\end{equation*}
. We have $\abs{D^\alpha m(\xi)}\leq C\abs{\xi}^{-\alpha}$\par
There are two standard techniques in proof of the Hormander-Mihlin multiplier theorem. The first one is
the smooth cutting of the multiplier into dyadic pieces.
\begin{lemma}[Lemma 6.5 in book]
    There is a non negative function $\phi\in C^\infty$ supported in the spherical shell $\{\xi: \frac{1}{2}\abs{\xi}<2\}$ such that
    \begin{equation*}
        \sum_{j\in\mathbb{Z}}\phi(2^{-j}\xi)=1\quad (\xi\neq 0)
    \end{equation*}
\end{lemma}
The second is an explicit formulation of the
well known fact that the regularity of the multiplier is translated into control of the size of kernel.
\begin{lemma}[Lemma 6.6 in book]
    Let $a=[\frac{n}{2}]+1$, and let $s$ be such that $a=\frac{n}{s}+\frac{1}{2}$ (so that $s\leq 2$). If $k\in L^2$ is
    such that $\hat{k}$ is of class $C^a$, then,
    \begin{equation*}
        \int_{\abs{x}>t}\abs{k(x)}dx\leq C_n t^{-\frac{1}{2}}\max_{\abs{\alpha}=a}{\norm{D^{\alpha}\hat{k}}_s}\quad (0<t<\infty)
    \end{equation*}
\end{lemma}
The details of proof of two lemmas and Hormander-Mihlin multiplier theorem is in book and the subsection \ref{details}.
% For example, when $n=1$
% \begin{align*}
%     m'(\xi)&=\lim_{h\to 0}\frac{m(\xi+h)-m(\xi)}{h}=\lim_{k\to 0}\frac{m(\xi+k\xi)-m(\xi)}{k\xi}\\
%     &=\lim_{k\to 0}\frac{(1+k)^{ib}-1}{k\xi}m(\xi)=i b \frac{m(\xi)}{\xi}
% \end{align*}
% Thus $\abs{m'(\xi)}=b \frac{\abs{m(\xi)}}{\abs{\xi}}=b\frac{\abs{\abs{\xi}^{ib}\abs{m(x')}}}{\abs{\xi}}=b\frac{\abs{m(x')}}{\abs{\xi}}\leq C\abs{\xi}^{-1}$.\par


\subsection{More precise estimation for more regular multiplier}
If we consider Mihlin multiplier theorem, the stronger condition \eqref{ieq: Mihlin condition} gives the smoothness of multipliers. And we have more precise estimation like Theorem 5.20 (i) in book.
\begin{theorem}[theorem 6.10 in book]
    Let $a$ be an integer such that $\frac{n}{2}<a\leq n$, and suppose that the multiplier $m(\xi)$ satisfies
    \eqref{ieq: Mihlin condition} for all $\abs{\alpha}\leq a$. Then, for every $q>\frac{n}{a}$, the operator $T_m$ satisfies
    \begin{equation*}
        (T_mf)^\#(x)\leq C_qM_qf(x)\quad (f\in\cup_{1<p<\infty}L^p)
    \end{equation*}
\end{theorem}

\subsection{Some properties of multipliers}
\begin{theorem}\label{dual of multiplier}
    $m$ is a multiplier for $L^p$ if and only if it is a multiplier for $L^{p'}$. And the norm of operator are identical.
\end{theorem}
\begin{proof}
    I want to use:
    \begin{align*}
        \int T_mf(x)\overline{g(x)}dx & =\int (\int m(y)\hat{f}(y) e^{2\pi i x\cdot y}d y)\overline{g(x)}dx                      \\
                                      & =\iint m(y)\hat{f}(y) e^{2\pi i x\cdot y}\overline{g(x)}d y d x                          \\
                                      & =\iint m(y)(\int f(z)e^{-2\pi i y\cdot z}d z) e^{2\pi i x\cdot y}\overline{g(x)}d y d x  \\
                                      & =\iiint m(y)f(z)e^{-2\pi i y\cdot z}e^{2\pi i x\cdot y}\overline{g(x)}d z d y d x        \\
                                      & =\iint m(y)f(z)e^{-2\pi i y\cdot z}\overline{(\int e^{-2\pi i x\cdot y}{g(x)}dx)}d z d y \\
                                      & =\iint m(y)f(z)e^{-2\pi i y\cdot z}\overline{\hat{g}(y)}d z d y                          \\
                                      & =\int f(z)\overline{(\int \overline{m(y)}\hat{g}(y)e^{2\pi i y\cdot z}d y)}d z           \\
                                      & =\int f(z)\overline{T_{\bar{m}}g(z)}d z                                                  \\
    \end{align*}
    By dual of $L^p$ and Hahn-Banach theorem,
    \begin{align*}
        \norm{T_m}= & \sup_{\norm{f}_p\leq 1}\norm{T_mf}_p                                                                                                                          \\
        =           & \sup_{\norm{f}_p\leq 1}\sup_{\norm{g}_q\leq 1}\int T_mf\bar{g}=\sup_{\norm{f}_p\leq 1}\sup_{\norm{g}_q\leq 1}\int f\overline{T_{\bar{m}}g}=\norm{T_{\bar{m}}}
    \end{align*}
    Thus $m$ is a multiplier for $L^p$ if and only if $\bar{m}$ is a multiplier for $L^{p'}$. And the norm of operator are identical. And $\bar{m}$ is a multiplier for $L^{p'}$ implies $m$ is a multiplier for $L^{p'}$.
    And their norms are equal.
\end{proof}
\begin{theorem}
    If $m$ is a multiplier for $L^p$, then $m$ is a multiplier for $L^q$ with $p'<q<p$ or $p<q<p'$.
\end{theorem}
This theorem is by theorem \ref{dual of multiplier} and Marcinkiewicz' interpolation theorem. If we use Riesz-Thorin interpolation
theorem, we can get estimation $\norm{T_m}_{p,p}\geq\norm{m}_\infty$.\par
\begin{theorem}
    The multipliers for $L^p$ form a subalgebra of $L^\infty(\mathbb{R}^n)$ which is invariant under
    translations, rotations and dilations.
\end{theorem}


\subsection{Details of proof and errata}\label{details}
\begin{note}[proof of lemma 6.1]
    Since the constant $C_p$ depends only on $p$, $n$ and on the constants $\norm{\hat{K}}_{\infty}$ and $B_K$ of
    the kernel, and $\norm{\hat{K}}_\infty$ and $B_K$ is bounded by hypothesis a) and b), $T_Nf=k_N*f$ are uniformly bounded.
\end{note}
\begin{errata}[P209]
    $k_N=k\chi_{\{x:\frac{1}{N}\leq \abs{x}\leq N\}}$
\end{errata}
\begin{note}[proof of norm of $T_m$ as an operator from $L^2$ to $L^2$ is $\norm{m}_\infty$]
    $T_m\leq\norm{m}_\infty$ is easy. To prove $T_m\geq\norm{m}_\infty$, by Plancherel's theorem, this is equivalent to
    $\norm{\hat{Tf}}_2\geq\norm{m}_\infty\norm{\hat{f}}_2$ for some $\hat{f}$. For any $\epsilon>0$, let $A_\epsilon=\{x\in [0,1]:m(x)>\norm{m}_\infty-\epsilon\}$, $\hat{f}(\xi)=\chi_{A_\epsilon}$.
    $\hat{Tf}=m\hat{f}>(\norm{m}_\infty-\epsilon)\hat{f}$ hence $\norm{\hat{Tf}}_2>(\norm{m}_\infty-\epsilon)\norm{\hat{f}}_2$. Thus $\norm{\hat{Tf}}_2\geq\norm{m}_\infty\norm{\hat{f}}_2$.
\end{note}
\begin{note}[proof of lemma 6.6 in book]
    The proof of lemma 6.6 is bad formatted. We give the complete proof of lemma 6.6.\par
    First we have $\abs{x}^a\leq (\sqrt{n\max_i{x_i}^2})^a\leq n^{\frac{a}{2}}max_i \abs{x_i}^a\leq n^{\frac{a}{2}}(\sum_i \abs{x_i}^a)$.\par
    By $\abs{x}^a\leq n^{\frac{a}{2}}(\sum_i \abs{x_i}^a)$, Holder inequality and Minkowski inequality for $s'\geq 2$:
    \begin{align*}
        \int_{\abs{x}>t}^{}{\abs{k(x)}dx} & = \int_{\abs{x}>t}^{}{\frac{\sum_{j=1}^n(\abs{x_j}^a)\abs{k(x)}}{\sum_{j=1}^n(\abs{x_j}^a)}dx}                                     \\
                                          & \leq \int_{\abs{x}>t}^{}{\frac{C\sum_{j=1}^n(\abs{x_j}^a)\abs{k(x)}}{\abs{x}^a}dx}                                                 \\
                                          & \leq C(\int_{}(\sum_{j=1}^n(\abs{x_j^ak(x)})^{s'})dx)^{\frac{1}{s'}}  (\int_{\abs{x}>t}^{} \frac{1}{\abs{x}^{as}}dx)^{\frac{1}{s}} \\
                                          & \leq C\sum_{j=1}^n(\int_{}\abs{x_j^ak(x)}^{s'}dx)^{\frac{1}{s'}}  (\int_{\abs{x}>t}^{} \frac{1}{\abs{x}^{as}}dx)^{\frac{1}{s}}     \\
    \end{align*}
    Notice $\frac{2}{3}\leq s\leq 2$, $n+\frac{1}{3}\leq as\leq n+1$. Thus $\int_{\abs{x}>t}^{} \frac{1}{\abs{x}^{as}}dx$ converges and
    $(\int_{\abs{x}>t}^{} \frac{1}{\abs{x}^{as}}dx)^{\frac{1}{s}}=Ct^{-\frac{1}{2}}$.\par
    Let $1\leq p\leq 2$ and $q$ is the dual exponent if $p$, the Hausdorff-Young inequality is:
    \begin{equation*}
        (\sum\abs{a_n}^q)^\frac{1}{q} \leq(\frac{1}{2\pi}\int_{0}^{2\pi}\abs{f(\theta)}^p d\theta)^\frac{1}{p}
    \end{equation*}
    and its dual:
    \begin{equation*}
        (\frac{1}{2\pi}\int_{0}^{2\pi}\abs{f(\theta)}^q d\theta)^\frac{1}{q}\leq(\sum\abs{a_n}^p)^\frac{1}{p}
    \end{equation*}
    But here it uses the analog for the Fourier transform (Corollary 2.6 in Chapter 2 in stein's \emph{Functional Analysis}):
    \begin{theorem}\label{Hausdorff-Young inequality}
        If $1\leq p\leq 2$ and $\frac{1}{p}+\frac{1}{q}=1$, then the Fourier transform $T$ has a unique extension to a bounded map from $L^p$ to $L^q$, with
        $\norm{T(f)}_q\leq\norm{f}_p$
    \end{theorem}
    By Theorem \ref{Hausdorff-Young inequality} and property of Fourier transform.
    \begin{align*}
        (\int_{}\abs{(x_j^ak(x))}^{s'}dx)^{\frac{1}{s'}} & \leq (\int_{}\abs{(x_j^ak(x))^\wedge}^{s}dx)^{\frac{1}{s}}                                 \\
                                                         & =(\int_{}\abs{(-2\pi i)^{-a}D_j^a\hat{k}(\xi)}^{s}dx)^{\frac{1}{s}}=C\norm{D_j^a\hat{k}}_s
    \end{align*}
    Thus
    \begin{equation*}
        \int_{\abs{x}>t}^{}{\abs{k(x)}dx}\leq Ct^{-\frac{1}{2}}\sum_{j=1}^n{\norm{D_j^a\hat{k}}_s}
    \end{equation*}
    Observe that $\sum_{j=1}^n{\norm{D_j^a\hat{k}}_s}\leq n\max_j{\norm{D_j^a\hat{k}}_s}\leq n\max_{\abs{\alpha}=a}{\norm{D^\alpha\hat{k}}_s}$. We finally prove that:
    \begin{equation*}
        \int_{\abs{x}>t}^{}{\abs{k(x)}dx}\leq Ct^{-\frac{1}{2}}\max_{\abs{\alpha}=a}{\norm{D^\alpha\hat{k}}_s}
    \end{equation*}
\end{note}


\begin{errata}[P212]
    By Leibnitz rule
    \begin{equation*}
        D^\alpha m_j(\xi)=\sum_{\alpha=\beta+\gamma}\binom{\abs{\alpha}}{\abs{\beta}} D^\beta m(\xi)2^{-j\abs{\gamma}}D^\gamma\phi(2^{-j}\xi)
    \end{equation*}
    with $\abs{\alpha}=\alpha_1+\alpha_2+\cdots+\alpha_n$
\end{errata}
\begin{errata}[P212]
    $m=\sum_j m_j$ and at most two $m_j$ can be non zero at any point. We have
    $\sum_j\abs{\hat{k_j}(\xi)}=\sum_j\abs{m_j(\xi)}\leq 2\norm{m}_\infty$
\end{errata}
\begin{errata}[P214]
    $\abs{(2\pi y)^\gamma}\leq C\abs{y}^{\abs{\gamma}}$
\end{errata}
\begin{note}[proof of theorem 6.3 in book]
    By Leibnitz rule we have
    \begin{equation*}
        D^\alpha m_j(\xi)=\sum_{\alpha=\beta+\gamma}\binom{\abs{\alpha}}{\abs{\beta}} D^\beta m(\xi)2^{-j\abs{\gamma}}D^\gamma\phi(2^{-j}\xi)
    \end{equation*}
    Since for each $\gamma$, $\abs{D^\gamma\phi(2^{-j}\xi)}$ is uniformly bounded and there are finite choice of $\gamma$, there is a
    constant $C$ with $\abs{D^\gamma\phi(2^{-j}\xi)}\leq C$ for all $\xi$ and $\gamma$. Notice $\binom{\abs{\alpha}}{\abs{\beta}}$ is also bounded. Thus
    \begin{equation*}
        \abs{ D^\alpha m_j(\xi)}\leq C'\sum_{\abs{\beta}\leq \abs{\alpha}}\abs{D^\beta m(\xi)2^{-j\abs{\gamma}}}=C'2^{-j\abs{\alpha}}\sum_{\abs{\beta}\leq \abs{\alpha}}\abs{D^\beta m(\xi)2^{j\abs{\beta}}}
    \end{equation*}
    Now we estimate the norm of $D^\alpha m_j(\xi)$. Notice $\supp(m_j)\subset\{\xi:2^{j-1}\leq \abs{\xi}\leq 2^{j+1}\}$. Thus the support of $D^\alpha m_j(\xi)$ is also in $\{\xi:2^{j-1}\leq \abs{\xi}\leq 2^{j+1}\}$. Thus
    only need to integrate each $D^\beta m(\xi)$ on $\{\xi:2^{j-1}\leq \abs{\xi}\leq 2^{j+1}\}$. By separate the region of integral:
    \begin{align*}
             & (\int_{2^{j-1}\leq \abs{\xi}\leq 2^{j+1}}\abs{D^\beta m(\xi)}^s d\xi)^{\frac{1}{s}}                                                                                   \\
        \leq & (\int_{2^{j-1}\leq \abs{\xi}\leq 2^{j}}\abs{D^\beta m(\xi)}^s d\xi)^{\frac{1}{s}}+  (\int_{2^{j}\leq \abs{\xi}\leq 2^{j+1}}\abs{D^\beta m(\xi)}^s d\xi)^{\frac{1}{s}}
    \end{align*}
    By Holder inequality and hypothesis \eqref{hypothesis of H-M multiplier theorem}, the first integral on right hand side is:
    \begin{align*}
             & (\int_{2^{j-1}\leq \abs{\xi}\leq 2^{j}}\abs{D^\beta m(\xi)}^s d\xi)^{\frac{1}{s}}                                                                          \\
        \leq & (\int_{2^{j-1}\leq \abs{\xi}\leq 2^{j}}\abs{D^\beta m(\xi)}^2 d\xi)^{\frac{1}{2}}(\int_{2^{j-1}\leq \abs{\xi}\leq 2^{j}} 1 d\xi)^{\frac{1}{s}-\frac{1}{2}} \\
        \leq & C_1 (2^{j-1})^{-\abs{\beta}}(2^{j-1})^\frac{n}{2} C_2(2^{j n}-2^{(j-1)n})^{\frac{1}{s}-\frac{1}{2}}                                                        \\
        =    & C_1C_2 (2^{j-1})^{\frac{n}{2}-\abs{\beta}}(2^{j n}(1-2^{-n}))^{\frac{1}{s}-\frac{1}{2}}                                                                    \\
        =    & C (1-2^{-n})^{\frac{1}{s}-\frac{1}{2}}(2^{-1})^{\frac{n}{2}-\abs{\beta}}(2^{j})^{\frac{n}{2}-\abs{\beta}}(2^{j n})^{\frac{1}{s}-\frac{1}{2}}               \\
        \leq & C'(2^{j})^{\frac{n}{2}-\abs{\beta}}(2^{j n})^{\frac{1}{s}-\frac{1}{2}}
    \end{align*}
    Using the same argument for the second integral on right hand side:
    \begin{align*}
        (\int_{2^{j}\leq \abs{\xi}\leq 2^{j+1}}\abs{D^\beta m(\xi)}^s d\xi)^{\frac{1}{s}} & \leq  C_1 (2^{j})^{-\abs{\beta}}(2^{j})^\frac{n}{2} C_2(2^{(j+1)n}-2^{j n})^{\frac{1}{s}-\frac{1}{2}}      \\
                                                                                          & =C (2^{n}-1)^{\frac{1}{s}-\frac{1}{2}}(2^{j})^{\frac{n}{2}-\abs{\beta}}(2^{j n})^{\frac{1}{s}-\frac{1}{2}} \\
                                                                                          & \leq C'(2^{j})^{\frac{n}{2}-\abs{\beta}}(2^{j n})^{\frac{1}{s}-\frac{1}{2}}
    \end{align*}
    Thus
    \begin{equation*}
        (\int_{2^{j-1}\leq \abs{\xi}\leq 2^{j+1}}\abs{D^\beta m(\xi)}^s d\xi)^{\frac{1}{s}} \leq  C'(2^{j})^{\frac{n}{2}-\abs{\beta}}(2^{j n})^{\frac{1}{s}-\frac{1}{2}}=C'(2^{j})^{-\abs{\beta}+\frac{n}{s}}
    \end{equation*}
\end{note}
\begin{note}[proof of theorem 6.3 in book]
    By Plancherel's theorem:
    \begin{equation*}
        \norm{T^Nf-T_mf}_2=\norm{(m-m^N)\hat{f}}_2=\norm{(m-m^N)}_2\norm{\hat{f}}_2
    \end{equation*}
\end{note}
Thus $T^Nf\to T_mf$ in $L^2$ for every $f\in L^2$.\par
Generally, $L^p$ convergence does not implies pointwise convergence. But there is a subsequence converges pointwise. Thus we have:
\begin{equation*}
    \liminf_{N\to \infty}\abs{T^Nf(x)-T_mf(x)}=0\quad a.e.
\end{equation*}
% \begin{align*}
%     \norm{T_m f(x)}_p \leq\norm{T^N f(x)-T_mf(x)}_p+\norm{T^N f(x)}_p\leq \norm{T^N f(x)-T_mf(x)}_p+C_p\norm{f}_p
% \end{align*}
By triangular inequality:
\begin{align*}
    (\int\abs{T_m f(x)}^p)^\frac{1}{p} \leq & (\int\abs{T^N f(x)-T_mf(x)}^p)^\frac{1}{p}+\int\abs{T^N f(x)}^p \\
    \leq                                    & (\int\abs{T^N f(x)-T_mf(x)}^p)^\frac{1}{p}+C_p\norm{f}_p
\end{align*}
Take $\liminf$ on both side and use Fatou's Lemma:
\begin{align*}
    (\int\abs{T_m f(x)}^p)^\frac{1}{p}\leq & \liminf(\int\abs{T^N f(x)-T_mf(x)}^p)^\frac{1}{p}+C_p\norm{f}_p \\
    \leq                                   & (\int\liminf\abs{T^N f(x)-T_mf(x)}^p)^\frac{1}{p}+C_p\norm{f}_p \\
    =                                      & C_p\norm{f}_p
\end{align*}

\begin{note}[proof of theorem 6.3 in book]\label{Leibnitz rule estimate for tilde m}
    By Leibnitz rule:
    \begin{align*}
        \abs{D^\alpha\tilde{m_j}(\xi)} & =\abs{\sum_{\alpha=\beta+\gamma}\binom{\abs{\alpha}}{\abs{\beta}} D^\beta m_j(\xi)D^\gamma(e^{-2\pi i y\cdot\xi}-1)}                                            \\
                                       & \leq C\sum_{\alpha=\beta+\gamma}\abs{D^\beta m_j(\xi)}\abs{D^\gamma(e^{-2\pi iy\cdot\xi}-1)}                                                                    \\
                                       & \leq C'\sum_{\alpha=\beta+\gamma}\abs{y}2^{j(1-\abs{\gamma})}\abs{D^\beta m_j(\xi)}                                                                             \\
                                       & \leq C'\sum_{\alpha=\beta+\gamma}\abs{y}2^{j(1-\abs{\gamma})}   C''2^{-j\abs{\beta}}\sum_{\abs{\beta'}\leq \abs{\beta}}\abs{D^{\beta'} m(\xi)2^{j\abs{\beta'}}} \\
                                       & \leq C\sum_{\alpha=\beta+\gamma}\abs{y}2^{j(1-\abs{\alpha})} \sum_{\abs{\beta'}\leq \abs{\beta}}\abs{D^{\beta'} m(\xi)2^{j\abs{\beta'}}}                        \\
                                       & \leq C\sum_{\abs{\beta}\leq a}\abs{y}2^{j(1-a)} \sum_{\abs{\beta'}\leq \abs{\beta}}\abs{D^{\beta'} m(\xi)2^{j\abs{\beta'}}}                                     \\
                                       & \leq C\sum_{\abs{\beta}\leq a}\abs{y}2^{j(1-a)} \sum_{\abs{\beta'}\leq {a}}\abs{D^{\beta'} m(\xi)2^{j\abs{\beta'}}}                                             \\
                                       & = Ca\abs{y}2^{j(1-a)} \sum_{\abs{\beta'}\leq {a}}\abs{D^{\beta'} m(\xi)2^{j\abs{\beta'}}}                                                                       \\
    \end{align*}
    Thus
    \begin{align*}
        \sup_{\abs{\alpha}=a}\norm{D^\alpha\tilde{m_j}}_s & \leq C\abs{y}2^{j(1-a)} \sum_{\abs{\beta'}\leq {a}}2^{j\abs{\beta'}}\norm{D^{\beta'} m(\xi)}_s        \\
                                                          & \leq C\abs{y}2^{j(1-a)} \sum_{\abs{\beta'}\leq {a}}2^{j\abs{\beta'}}C2^{j(-\abs{\beta'}+\frac{n}{s})} \\
                                                          & \leq C'\abs{y}2^{j(1-a)} \sum_{\abs{\beta'}\leq {a}}2^{j\frac{n}{s}}                                  \\
                                                          & \leq C' a\abs{y}2^{j(1-a+\frac{n}{s})}                                                                \\
    \end{align*}


\end{note}
\begin{note}[proof of theorem 6.10 in book]
    We denote $t_j=2^j\abs{y}$
    \begin{align*}
             & \int_{\abs{x}>2\abs{y}}{\abs{k^N(x-y)-k^N(x)}\abs{f(x)}dx}                                                                                                                                \\
        =    & \sum_{j=1}^{\infty}\int_{t_j<\abs{x}\leq 2t_j}{\abs{k^N(x-y)-k^N(x)}\abs{f(x)}dx}                                                                                                         \\
        \leq & \sum_{j=1}^{\infty}(\int_{t_j<\abs{x}\leq 2t_j}{\abs{k^N(x-y)-k^N(x)}^{q'}dx})^\frac{1}{q'}(\int_{t_j<\abs{x}\leq 2t_j}{\abs{f(x)}^q dx})^\frac{1}{q}                                     \\
        \leq & \sum_{j=1}^{\infty}(\int_{t_j<\abs{x}}{\abs{k^N(x-y)-k^N(x)}^{q'}dx})^\frac{1}{q'}(\int_{t_j<\abs{x}\leq 2t_j}{\abs{f(x)}^q dx})^\frac{1}{q}                                              \\
        \leq & (\sum_{j=1}^{\infty}Ct_{j}^{-\epsilon-\frac{n}{q}}\abs{y}^{\epsilon}\abs{B(0,2t_j)}^{\frac{1}{q}})\sup_j{(\frac{1}{\abs{B(0,2t_j)}}\int_{\abs{x}\leq 2t_j}{\abs{f(x)}^q dx})^\frac{1}{q}} \\
        \leq & C(\sum_{j=1}^{\infty}t_{j}^{-\epsilon-\frac{n}{q}}\abs{y}^{\epsilon}{t_j}^{\frac{n}{q}})M_q f(0)                                                                                          \\
        =    & C\abs{y}^{\epsilon}(\sum_{j=1}^{\infty}t_{j}^{-\epsilon})M_q f(0)                                                                                                                         \\
        =    & C\abs{y}^{\epsilon}\abs{y}^{-\epsilon}(\sum_{j=1}^{\infty}2^{-j\epsilon})M_q f(0)                                                                                                         \\
        =    & C'M_q f(0)                                                                                                                                                                                \\
    \end{align*}
    Since we prove that $(T_mf)^\#(0)\leq A_n\sup_{\epsilon>0}\epsilon^{-n}\int_{\abs{x}\leq\frac{\epsilon}{2}}\abs{f(y)-I_\epsilon}dy$ and using lemma 5.11 in book:
    \begin{align*}
             & \epsilon^{-n}\int_{\abs{x}<\frac{\epsilon}{2}}{\abs{Tf(x)-I_\epsilon}dx}                                    \\
        \leq & C_qM_qf(0)+\epsilon^{-n}\iint_{2\abs{x}<\epsilon<\abs{y}}{\abs{K(x-y)-K(-y)}\abs{f(y)}dxd y}                \\
        \leq & C_qM_qf(0)+\epsilon^{-n}\int_{2\abs{x}<\epsilon}{(\int_{2\abs{x}<\abs{y}}\abs{K(y-x)-K(y)}\abs{f(y)}d y)dx} \\
        \leq & C_qM_qf(0)+\epsilon^{-n}\int_{2\abs{x}<\epsilon}{(C'M_q f(0))dx}                                            \\
        \leq & C_qM_qf(0)+C''M_q f(0)                                                                                      \\
        \leq & CM_q f(0)
    \end{align*}
    Thus  $(T^Nf)^\#(0)\leq CM_qf(0)$.
\end{note}
\begin{note}[proof of (6.11) in book]
    First we prove a variant of Lemma 6.6:
    \begin{equation}\label{variant of lemma 6.6}
        (\int_{\abs{x}>t}\abs{k(x)}^{q'}dx)^\frac{1}{q'}\leq C t^{\epsilon-\frac{n}{q}}\max_{\abs{\alpha}=a}{\norm{D^{\alpha}\hat{k}}_s}\quad (0<t<\infty)
    \end{equation}
    The proof is entirely similar with lemma 6.6, but Holder's inequality is applied
    in the form: $\norm{\cdot}_{q'}\leq \norm{\cdot}_{r}\norm{\cdot}_{s'}$, with $\frac{1}{r}=\frac{1}{s}-\frac{1}{q}$. Notice $-ar+n-1<-1$ by $a-\frac{n}{s}=\epsilon>-\frac{1}{q}$, the second integral on right hand side converges.
    \begin{align*}
        (\int_{\abs{x}>t}\abs{k(x)}^{q'}dx)^\frac{1}{q'} & \leq C(\int_{}(\sum_{j=1}^n(\abs{x_j^ak(x)})^{s'})dx)^{\frac{1}{s'}}  (\int_{\abs{x}>t}^{} \frac{1}{\abs{x}^{ar}}dx)^{\frac{1}{r}} \\
                                                         & \leq C\sum_{j=1}^n(\int_{}\abs{x_j^ak(x)}^{s'}dx)^{\frac{1}{s'}}  (t^{n-a r})^{\frac{1}{r}}                                        \\
                                                         & \leq C\max_{\abs{\alpha}=a}{\norm{D^\alpha\hat{k}}_s}  (t^{\frac{n}{r}-a})                                                         \\
    \end{align*}
    By $\frac{n}{r}-a=\frac{n}{s}-\frac{n}{q}-a=-\epsilon-\frac{n}{q}$, we have proof inequality \eqref{variant of lemma 6.6}.\par
    Now we prove:
    \[
        (\int_{\abs{x}>2t}\absl{k_j(x-y)-k_j(x)}^{q'}dx)^\frac{1}{q'}\leq\left\{
        \begin{aligned}
            Ct^{-\epsilon-\frac{n}{q}}2^{-j\epsilon}\quad (2^j\abs{y}\geq 1) \\
            C t^{-\epsilon-\frac{n}{q}} \abs{y}2^{j(1-\epsilon)}\quad (2^j\abs{y}<1)
        \end{aligned}
        \right.
    \]

    The first inequality is by inequality \eqref{variant of lemma 6.6} and inequality (6.7) in book:
    \begin{equation*}
        \norm{D^{\alpha}m_j}_s\leq C 2^{j(-\abs{\alpha}+\frac{n}{s})}\quad (\abs{\alpha}\leq a; 1\leq s\leq 2)
    \end{equation*}
    we have
    \begin{equation}\label{variant 2 of lemma 6.6}
        (\int_{\abs{x}>t}\abs{k_j(x)}^{q'}dx)^\frac{1}{q'}\leq Ct^{-\epsilon-\frac{n}{q}}2^{-j\epsilon}
    \end{equation}


    The proof of the second inequality is similar with (6.9) in book. By the same argument as
    above, we have
    \begin{equation*}
        (\int_{\abs{x}>t}\absl{\tilde{k_j}(x)}^{q'}dx)^\frac{1}{q'} \leq C t^{-\epsilon-\frac{n}{q}} \max_{\abs{\alpha}=a}{\norm{D^\alpha\hat{\tilde{k_j}}}_s}
    \end{equation*}
    By $\hat{\tilde{k}}=\tilde{m}$ and inequality (6.9) in book
    \begin{equation*}
        \sup_{\abs{\alpha}=a}\norm{D^{\alpha}\tilde{m_j}}_s\leq C \abs{y}2^{j(1-a+\frac{n}{s})}=C \abs{y}2^{j(1-\epsilon)}\quad (1\leq s\leq 2; 2^j\abs{y}<1)
    \end{equation*}
    , we have:
    \begin{equation}\label{variant of 6.9}
        (\int_{\abs{x}>t}\absl{\tilde{k_j}(x)}^{q'}dx)^\frac{1}{q'} \leq C t^{-\epsilon-\frac{n}{q}} \abs{y}2^{j(1-\epsilon)}\quad (1\leq s\leq 2; 2^j\abs{y}<1)
    \end{equation}

    Now by inequality \eqref{variant 2 of lemma 6.6} and inequality \eqref{variant of 6.9} we have:
    \[
        (\int_{\abs{x}>2t}\absl{k_j(x-y)-k_j(x)}^{q'}dx)^\frac{1}{q'}\leq\left\{
        \begin{aligned}
            Ct^{-\epsilon-\frac{n}{q}}2^{-j\epsilon}\quad (2^j\abs{y}\geq 1) \\
            C t^{-\epsilon-\frac{n}{q}} \abs{y}2^{j(1-\epsilon)}\quad (2^j\abs{y}<1)
        \end{aligned}
        \right.
    \]
    Finally, by summing with different part of index, (6.11) is an easy consequence of above inequality.
\end{note}
\begin{note}[proof of (6.14) in book]
    By definition of subalgebra, we only need to check that $T_{am_1+bm_2}$, $T_{m_1m_2}$ are bounded. These are easy since $T_m$ is linear on $m$ and $T_{m_1m_2}=T_{m_1}T_{m_2}$. Suppose the affine transformations $A(x)=h+L(x)$ where $L$ is linear transformation. {\color{blue} I don't know why the following holds:
            \begin{equation*}
                (f\circ A)^\wedge(\xi)=e^{2\pi i h\cdot\tilde{L}(\xi)}\abs{\det{L}}^{-1}\hat{f}(\tilde{L}(\xi))
            \end{equation*}
            with $\tilde{L}=(L^*)^{-1}$ and how this implies multipliers are invariant under transformation $A$}.
\end{note}

\end{document}

